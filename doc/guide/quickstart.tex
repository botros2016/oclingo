\section{Quickstart}

In ASP it is custom to provide a \emph{uniform}
problem definition~\cite{martru99a,niemela99a,schlipf95a}.
Following this methodology we separate the encoding
from an instance of the following problem:
given a (directed) graph, decide whether each node can be assigned
one of~$N$ colors such that any pair of adjacent nodes is colored differently.
Note that this problem is NP-complete for~$N\geq 3$
(see, e.g.,~\cite{papadimitriou94a}),
and thus it seems unlikely that a worst-case polynomial algorithm
can be found.
In view of this,
it is convenient to reduce the particular problem to
a declarative problem solving paradigm like ASP,
where efficient off-the-shelf tools like \gringo\ and \clasp\
are ready to solve the problem reasonably well.
Such a reduction is now exemplified.

\subsection{Problem Instance}\label{subsec:color:instance}
\begin{figure}[tb]
\begin{center}
\begin{picture}(200,130)(40,30)
%
\put(80,75){\circle{16}}   \put(77.5,72){1}
\put(80,145){\circle{16}}  \put(77.5,142){3}
\put(140,25){\circle{16}}  \put(137.5,22){4}
\put(140,110){\circle{16}} \put(137.5,107){6}
\put(200,75){\circle{16}}  \put(197.5,72){2}
\put(200,145){\circle{16}} \put(197.5,142){5}
%
\put(92,75){\line(1,0){96}}  % 1 -> 2
\put(80,87){\line(0,1){46}}  % 1 -> 3
\put(91,69){\line(1,-1){39}} % 1 -> 4
%
\put(189,69){\line(-1,-1){39}} % 2 -> 4
\put(200,87){\line(0,1){46}}   % 2 -> 5
\put(190,81){\line(-3,2){39}}  % 2 -> 6
%
\qbezier(71,138)(15,57)(128,25) 
\put(127.5,25){\line(4,-1){2}} % 3 -> 4
\put(92,145){\line(1,0){96}}   % 3 -> 5
\put(129,113){\line(-3,2){39}} % 6 -> 3
%
\put(188,145){\line(-1,0){96}}  % 5 -> 3
\qbezier(209,138)(265,57)(152,25) 
\put(152.5,25){\line(-4,-1){2}} % 5 -> 4
%
\put(151,113){\line(3,2){39}}  % 6 -> 5
\end{picture}
\end{center}
\caption{A Directed Graph with Six Nodes and Nine Edges.\label{fig:graph_intro}}
\end{figure}

%%% Local Variables: 
%%% mode: latex
%%% TeX-master: "../guide"
%%% End: 


We consider directed graphs specified via facts over predicates
\pred{node}/$1$ and \pred{edge}/$2$.%
The predicate \pred{color}/$1$ denotes the colors that shall be used to color the graph.
Figure~\ref{fig:graph_intro} depicts the graph specified by the following set of facts:
%
\lstinputlisting{examples/graph_intro.lp}

\subsection{Problem Encoding}\label{subsec:color:encoding}

We now proceed by encoding $N$-coloring via non-ground rules that are
independent of particular instances.
Typically, an encoding consists of a \emph{Generate}, a \emph{Define},
and a \emph{Test} part~\cite{lifschitz02a}.
As $N$-Coloring has a rather simple pattern, the following encoding does
not contain any Define part:
%
\lstinputlisting{examples/color.lp}
%
The Generate rule in Line~2 makes use of the \const{count} aggregate
(cf.\ Section~\ref{subsec:gringo:aggregate}).
For our example graph and~\const{1} substituted for~\var{X},
we obtain the following ground rule:%
\marginlabel{%
  The full ground program is obtained by invoking:\\
  \code{\mbox{~}gringo -t \textbackslash\\
    \mbox{~}examples/color.lp \textbackslash\\
    \mbox{~}examples/graph.lp}}
%
\begin{lstlisting}[numbers=none]
1 { color(1,red), color(1,green), color(1,blue) } 1.
\end{lstlisting}
%
The \code{1} on the left hand side (right hand side) specifies that at least (at most) on atom within the curly brackets has to be true
that is~\code{\pred{node}(\const{1})} must have exactly one color.
Note that~\code{\pred{node}(\const{1})} has been removed from the body,
as it is derived via a corresponding fact and \code{color(1,C)} has been expanded using 
the colors \code{red}, \code{green}, and \code{blue} given by \pred{color}/$1$ specified in the instance.
Similar ground instances are obtained for the other nodes~\const{2} to~\const{6}.
With these six rules alone we already have a coloring the graph,
which does not yet ensure that two adjacent nodes have different colors.
This property is asserted by the rule in line four in the test part.
For each instance of \pred{edge}/2,
we obtain~\code{n} ground instances of the integrity constraint in Line~6,
prohibiting that the same color~\var{C} is assigned to the adjacent nodes.
We get the following ground instances due to \code{\pred{edge}(\const{1},\const{2})}:
%
\begin{lstlisting}[numbers=none]
:- color(1,red), color(2,red).
:- color(1,green), color(2,green).
:- color(1,blue), color(2,blue).
\end{lstlisting}
%
Again note that \code{\pred{edge}(\const{1},\const{2})},
derived via a fact, has been removed from the body.

\subsection{Problem Solution}\label{subsec:color:solution}
\begin{figure}[tb]
\begin{center}
\begin{picture}(200,130)(40,30)
%
\put(80,75){\color{Peach}\circle*{16}}            \put(77.5,72){1}
\put(80,145){\color{Green}\circle*{16}}           \put(77.5,142){3}
\put(140,25){\color{CornflowerBlue}\circle*{16}}  \put(137.5,22){4}
\put(140,110){\color{CornflowerBlue}\circle*{16}} \put(137.5,107){6}
\put(200,75){\color{Green}\circle*{16}}           \put(197.5,72){2}
\put(200,145){\color{Peach}\circle*{16}}          \put(197.5,142){5}
%
\put(92,75){\vector(1,0){96}}  % 1 -> 2
\put(80,87){\vector(0,1){46}}  % 1 -> 3
\put(91,69){\vector(1,-1){39}} % 1 -> 4
%
\put(189,69){\vector(-1,-1){39}} % 2 -> 4
\put(200,87){\vector(0,1){46}}   % 2 -> 5
\put(190,81){\vector(-3,2){39}}  % 2 -> 6
%
\put(80,133){\vector(0,-1){46}}  % 3 -> 1
\qbezier(71,138)(15,57)(128,25) 
\put(127.5,25){\vector(4,-1){2}} % 3 -> 4
\put(92,145){\vector(1,0){96}}   % 3 -> 5
%
\put(130,30){\vector(-1,1){39}} % 4 -> 1
\put(150,30){\vector(1,1){39}}  % 4 -> 2
%
\put(188,145){\vector(-1,0){96}}  % 5 -> 3
\qbezier(209,138)(265,57)(152,25) 
\put(152.5,25){\vector(-4,-1){2}} % 5 -> 4
\put(190,139){\vector(-3,-2){39}} % 5 -> 6
%
\put(151,107){\vector(3,-2){39}} % 6 -> 2
\put(129,113){\vector(-3,2){39}} % 6 -> 3
\put(151,113){\vector(3,2){39}}  % 6 -> 5
%
\end{picture}
\end{center}
\caption{A $3$-Coloring for the Graph in Figure~\ref{fig:graph}.\label{fig:color}}
\end{figure}

%%% Local Variables: 
%%% mode: latex
%%% TeX-master: "../guide"
%%% End: 


Provided that a given graph is colorable with the colors given in the instance,
a solution can be read off an answer set of the program consisting
of the instance and the encoding.
For the graph in Figure~\ref{fig:graph_intro},
the following answer set can be computed:%
\marginlabel{%
  To find an answer set, invoke:\\
  \code{\mbox{~}gringo \textbackslash\\
        \mbox{~}examples/color.lp \textbackslash\\
        \mbox{~}examples/graph.lp |\rlap{\:\textbackslash}\\
        \mbox{~}clasp}\\
  or alternatively:\\
  \code{\mbox{~}clingo \textbackslash\\
        \mbox{~}examples/color.lp \textbackslash\\
        \mbox{~}examples/graph.lp}}
%
\begin{lstlisting}[numbers=none]
Answer: 1
... color(1,red)  color(2,green) color(3,green)
    color(4,blue) color(5,red)   color(6,blue)
\end{lstlisting}
%
Note that we have omitted the instances of \pred{color}/$1$, \pred{node}/$1$ 
and \pred{edge}/$2$ in order to emphasize the actual solution,
which is depicted in Figure~\ref{fig:color}.
Such output projection can also be specified within a logic program file by
using the declaratives \code{\#hide} and \code{\#show},
described in Section~\ref{subsec:gringo:meta}.
