\section{Errors and Warnings}\label{sec:error:warn}

This section explains the most frequent errors and warnings
related to inappropriate inputs or command line options that,
if they occur, lead to messages sent to the standard error stream.
The difference between errors and warnings is that the former
involve immediate termination,
while the latter are hints pointing at possibly corrupt input
that can still be processed further.
In the below description of errors (Section~\ref{subsec:error})
and warnings (Section~\ref{subsec:warn}),
we refrain from attributing them to a particular one among the tools
\gringo, \clasp, \clingo, and \iclingo,
in view of the fact that they share a number of functionalities.


\subsection{Errors}\label{subsec:error}

We start our description with errors that may be encountered during grounding,
where the following one indicates a syntax error in the input:
%
\begin{lstlisting}[numbers=none,escapechar=@]
syntax error on line @\textit{L}@ column @\textit{C}@ unexpected token: @\textit{T}@
\end{lstlisting}
%
To correct this error, please investigate the line~\code{\textit{L}}
and check whether something looks strange there
(like a missing period, an unmatched parenthesis, etc.).

The next error occurs if an input program is not level-restricted
(cf.\ Section~\ref{subsec:lambda}):
%
\begin{lstlisting}[numbers=none,escapechar=@]
the following rule cannot be grounded, \
weakly restricted variables: { @\textit{V}@, ... }
        @\textit{rule}@
\end{lstlisting}
%
Along with the error message, the \code{\textit{rule}} and the
name~\code{\textit{V}} of at least one variable causing the problem
are reported.
The first action to take usually consists of checking whether
variable~\code{\textit{V}} is actually in the scope of any atom
(in the positive body of \code{\textit{rule}}) that can bind it.%
\footnote{%
  Recall from Section~\ref{subsec:gringo:arith} and~\ref{subsec:gringo:comp}
  that a variable in the scope of a built-in arithmetic function may not be bound
  by a corresponding atom and that built-in comparison predicates do not bind
  any variable.}
If~\code{\textit{V}} is a local variable belonging to an atom~$A$ 
on the left-hand side of a condition (cf.\ Section~\ref{subsec:gringo:condition})
or to an aggregate (cf.\ Section~\ref{subsec:gringo:aggregate}),
an atom over some domain predicate might be included in a condition
to bind~\code{\textit{V}}.
In particular, if~$A$ itself is over a domain predicate,
the problem is often easily fixed by writing ``\code{$A$:$A$}.''
If the problem is not as simple,
then the predicates of all atoms that in principle may be used
to bind~\code{\textit{V}} are involved in positive recursion
through the predicate of some atom in the head of \code{\textit{rule}}.
In this case, the positive bodies of some of the rules subject to recursion
must be augmented with additional atoms over non-recursive predicates that
can bind variables, in order to eventually establish level-restrictedness.
As breaking recursion is a rather global matter, the reported
\code{\textit{rule}} is not necessarily the best place to tackle it.

The following error is related to conditions
(cf.\ Section~\ref{subsec:gringo:condition}):
%
\begin{lstlisting}[numbers=none,escapechar=@]
the following rule cannot be grounded, \
non domain predicates : { @\textit{p}@(...), ... }
        @\textit{rule}@
\end{lstlisting}
%
The problem is that an atom \code{\textit{p}(...)} such
that its predicate \code{\textit{p}/\!\textit{i}} is not a domain predicate
(cf.\ Section~\ref{subsec:strat}) is used on the right-hand side of a condition
within \code{\textit{rule}}.
The error is corrected by either removing the atom or
by replacing it with another atom over a domain predicate.

The next errors may occur within an arithmetic evaluation
(cf.\ Section~\ref{subsec:gringo:arith}):
%
\begin{lstlisting}[numbers=none,escapechar=@]
trying to convert string to int
trying to convert functionsymbol to int
\end{lstlisting}
%
It means that either a (symbolic) constant or a compound term
(over an uninterpreted function with non-zero arity)
has occurred in the scope of some built-in arithmetic function.
Unfortunately, no context information is provided yet,
which is left as an issue to future work
(cf.\ Section~\ref{sec:future}).

The below errors are related to built-in comparison predicates
(cf.\ Section~\ref{subsec:gringo:comp}):
%
\begin{lstlisting}[numbers=none,escapechar=@]
comparing different types
comparing function symbols
\end{lstlisting}
%
The first error indicates that one of the built-in comparison predicates
\code{<}, \code{<=}, \code{>}, and \code{>=}
has been applied to an integer or a (symbolic) constant
and a term of a different sort,
while the second error means that compound terms have occurred in such a comparison.
The mere reason for these errors is that the required functionalities
have not been implemented yet,
so that the errors are likely to disappear in the future
(cf.\ Section~\ref{sec:future}).

An error occurs if a constant~\code{\textit{c}}
ought to be replaced with a term~\code{\textit{t}}
containing~\code{\textit{c}}:
%
\begin{lstlisting}[numbers=none,escapechar=@]
cyclic constant definition
\end{lstlisting}
%
To resolve such an error, check the occurrences of \const{\#const}
in the input as well as \code{--const} or \code{-c} options
in the command line, and change them in a way such that a constant
that is to be replaced does not (transitively) occur in a term
to be inserted.

The next error may occur if a statement from the input language
of \iclingo\ (cf.\ Section~\ref{subsec:lang:iclingo})
is used in the input of \gringo, \clingo, or
\iclingo\ run non-incrementally
(via one of options \code{--text}, \code{--lparse}, \code{--aspils},
 or \code{--clingo}):
%
\begin{lstlisting}[numbers=none,escapechar=@]
A fixed number of incremental steps is needed to ground the program.
\end{lstlisting}
%
This problem is solved by providing a step number
via option \code{--ifixed}.

The following error may be reported by \iclingo:
%
\begin{lstlisting}[numbers=none,escapechar=@]
The following statement cant be used with the incremental interface:
        @\textit{rule}@
\end{lstlisting}
%
The \code{\textit{rule}} contains a \const{\#maximize},
\const{\#minimize}, or \const{\#compute} statement,
which are currently not supported within incremental computations.
Their support by \iclingo\ is an issue to future work
(cf.\ Section~\ref{sec:future}).

The next error may occur if \emph{ASPils}~\cite{gejaosscth08a}
output is requested via option \code{--aspils}:
%
\begin{lstlisting}[numbers=none,escapechar=@]
@\textit{e}@ not allowed in this normal form, \ 
please choose normal form @\textit{n}@.
\end{lstlisting}
%
The chosen normal form (in-between~\code{1} and~\code{7})
does not cover the input language expression~\code{\textit{e}},
and one among the normal forms~\code{\textit{n}} suggested
in the error message ought to be provided as an argument instead.

The following error message is issued by (embedded) \clasp:
%
\begin{lstlisting}[numbers=none,escapechar=@]
Input Error at line @\textit{L}@: Symbol table expected!
\end{lstlisting}
%
This error means that the input does not comply with \lparse's
numerical format~\cite{lparseManual}.
It is not unlikely that the input can be processed
by \gringo, \clingo, or \iclingo.

The next error indicates that input in
\lparse's numerical format~\cite{lparseManual} is corrupt:
%
\begin{lstlisting}[numbers=none,escapechar=@]
Input Error at line @\textit{L}@: Atom out of bounds
\end{lstlisting}
%
There is no way to resolve this problem.
If the input has been generated by \gringo, \clingo, or \iclingo,
please report the problem to the authors of this guide.

The following error message is issued by (embedded) \clasp:
%
\begin{lstlisting}[numbers=none,escapechar=@]
Input Error at line @\textit{L}@: Unsupported rule type!
\end{lstlisting}
%
It means that some rule type in \lparse's
numerical format~\cite{lparseManual} is not supported.
Most likely, the program under consideration contains
rules with disjunction in the head.

A similar error may occur with \clingo\ or \iclingo:
%
\begin{lstlisting}[numbers=none,escapechar=@]
sorry clasp cannot handle disjunctive rules!
\end{lstlisting}
%
The program under consideration contains rules with disjunction in the head,
which are currently not supported by \clasp,
but by \claspD~\cite{drgegrkakoossc08a}.
The integration of \clasp\ and \claspD\ is a subject to future work
(cf.\ Section~\ref{sec:future}).

All of the tools \gringo, \clasp, \clingo, and \iclingo\
try to expand incomplete (long) options to recognized ones.
Parsing command line options may nonetheless fail due to the following three reasons:
%
\begin{lstlisting}[numbers=none,escapechar=@]
unknown option: @\textit{o}@
ambiguous option: '@\!\textit{o}@' could be:
  @\textit{o1}@
  @\textit{o2}@
  ...
'@\!\textit{a}@': invalid value for Option '@\!\textit{o}@'
\end{lstlisting}
%
The first error means that a provided option~\code{\textit{o}}
could not be expanded to one that is recognized,
while the second error expresses that the result of expanding~\code{\textit{o}}
is ambiguous.
Finally, the third error occurs if a provided argument~\code{\textit{a}}
is invalid for option~\code{\textit{o}}.
In either case, option \code{--help} can be used to see 
the recognized options and their arguments.


\subsection{Warnings}\label{subsec:warn}

The following warnings may be raised by \gringo, \clingo, or \iclingo:
%
\begin{lstlisting}[numbers=none,escapechar=@]
Warning: @\textit{p}@/@\!\textit{i}@ is never defined.
Warning: sum with negative bounds in the body of a rule
Warning: multiple definitions of #const @\textit{c}@
\end{lstlisting}
%
The first one, stating that a predicate \code{\textit{p}/\!\textit{i}}
has occurred in some rule body, but not in the head of any rule,
might point at a mistyped predicate.
The second warning means that negative weights within a \const{sum} aggregate
(cf.\ Section~\ref{subsec:gringo:aggregate})
have been compiled away~\cite{siniso02a},
which is semantically delicate~\cite{ferraris05a}.
Thus, we strongly recommend to avoid negative weights
within a \const{sum} aggregate in the positive body of a rule.
Finally, the third warning recognizes that multiple \const{\#const}
statements over the same constant~\code{\textit{c}} are contained in the input,
but only the first of them will be considered.

The next four warnings may be issued by \iclingo:
\begin{lstlisting}[numbers=none,escapechar=@]
Warning: There are no #base, #cumulative or #volatile sections.
Warning: Option ifixed will be ignored!
Warning: There are statements not within a #base, \
         #cumulative or #volatile section.
         These Statements are put into the #base section.
Warning: Classical negation is not handled correctly \
         in combination with the incremental output.
         You have to add rules like: :- a, -a. on your own! \
         (at least for now)
\end{lstlisting}
%
For omitting the first warning, one may use option \code{--clingo},
as the input program contains no statements particular to \iclingo\
(cf.\ Section~\ref{subsec:lang:iclingo}).
Similarly, option \code{--clingo} may be used to suppress the second warning,
stating that a provided option \code{--ifixed} is ignored in incremental mode
(if \iclingo\ is not run as a grounder 
 via one of options \code{--text}, \code{--lparse}, or \code{--aspils}).
The third warning states that rules have been inserted into the static
program part without a 	preceding \const{\#base} statement.
This warning can be avoided by ending an encoding with \const{\#base.}
and by supplying an instance file afterwards, as illustrated in Section~\ref{subsec:ex:block}.
The fourth warning refers to the fact that there currently is no
built-in support for classical negation (cf.\ Section~\ref{subsec:gringo:negation})
by \iclingo.
Hence, integrity constraints of the form ``$\code{:-}~A\code{,}\code{-}A\code{.}$''
have to be added explicitly by a user.


%%% Local Variables: 
%%% mode: latex
%%% TeX-master: "guide"
%%% End: 
