\section{Input Languages}\label{sec:language}

This section provides an overview of the input languages of
grounder \gringo, combined grounder and solver \clingo,
incremental grounder and solver \iclingo, and of solver \clasp.
The joint input language of \gringo\ and \clingo\ is detailed in
Section~\ref{subsec:lang:gringo}.
It is extended by \iclingo\ with a few directives described in Section~\ref{subsec:lang:iclingo}.
Finally, Section~\ref{subsec:lang:clasp} is dedicated to the inputs
handled by \clasp.


\subsection{Input Language of \gringo\ and \clingo}\label{subsec:lang:gringo}

The tool \gringo~\cite{gescth07a} is a grounder capable of translating
logic programs provided by users into equivalent ground programs.
The output of \gringo\ can be piped into solver \clasp~\cite{gekanesc07b},
which then computes answer sets.
System \clingo\ internally couples \gringo\ and \clasp, thus,
it takes care of both grounding and solving.
% The input of \gringo\ and \clingo\ is the same,
% but 
In contrast to \gringo\ outputting ground programs, 
\clingo\ returns answer sets.
Both \gringo\ and \clingo\ can handle safe input programs
(cf.\ Section~\ref{subsec:lambda}),
usually specified in one or more text files whose names are passed via the command line
in an invocation of either \gringo\ or \clingo.
We below provide a description of constructs belonging to the input language of \gringo\ and \clingo.


\subsubsection{Normal Programs and Integrity Constraints}\label{subsec:gringo:normal}

In the previous sections, we have already seen a number of normal programs.
Now also considering integrity constraints, we get the following basic rule types:

\begin{tabular}{rl@{}l}
\textbf{Fact:} & $A_0$&\code{.}
\\
\textbf{Rule:} & $A_0$& $\code{:-}~L_1\code{,}\dots\code{,}L_n$\code{.}
\\
\textbf{Integrity Constraint:} & & $\code{:-}~L_1\code{,}\dots\code{,}L_n$\code{.}
\end{tabular}
\\

\noindent
The head~$A_0$ of a fact or a rule is an \emph{atom} of the form
$p\code{(}t_1\code{,}\dots\code{,}t_m\code{)}$,
where $p$ is the name of some predicate, that is, a sequence of letters, digits and underscores 
starting with a lowercase letter,
and any $t_i$ is a term.%
\footnote{An atom~$p$ without arguments is simply a sequence of letters and digits
  (starting with a lowercase letter).
  For such an atom~$p$, parentheses after the name are skipped, e.g.,
  \pred{p42X} could be an atom without arguments.}
A \emph{term} $t$ starting with an uppercase letter followed by
a sequence of letters, digits and underscores (e.g., \var{X\_08x15}) is a variable name,
and integers are written as sequences of digits possibly preceded by ``\code{-}.''
In addition, a term can have the same syntactic structure as an atom
(e.g., \code{\pred{p}(\const{a},\const{1},\const{f}(\var{X}))}
 can be either an atom or a term, depending on where it occurs in a rule),
and functions can be nested within a term.
There are also built-in constructs
(cf.\ Section~\ref{subsec:gringo:arith} and~\ref{subsec:gringo:comp})
having particular representations.
Finally, any $L_j$ is a \emph{literal} of the form $A$ or $\code{not}~A$
for an atom~$A$.

In Section~\ref{subsec:semantics},
we have already provided a translation~$\psi$
from the head of a rule to a propositional formula.
By viewing an integrity constraint
$\code{:-}~\mathit{Body}\code{.}$ as a shorthand for $\bot~\code{:-}~\mathit{Body}\code{.}$,
we can now explain the semantics of integrity constraints by adding the following case:
\begin{itemize}
\item $\psi[\bot]=\bot$.
\end{itemize}
Note the role of integrity constraints is to eliminate answer set candidates.
In fact, given an integrity constraint~$\code{:-}~\mathit{Body}\code{.}$,
any answer set~$X$ is such that $X\not\models\phi[\mathit{Body}]$, or
in other words, some literal~$L$ in $\mathit{Body}$ must be unsatisfied w.r.t.~$X$.
Elaborate examples on the usage of facts, rules, and integrity constraints
will be provided in Section~\ref{sec:examples}.

\begin{newstuff}
\subsubsection{Anonymous Variables}
Variables that start with an underscore are so called \emph{anonymous variables}.
Each such variable is treated like a distinct variable name.
They are especially helpful to mark places where we do not care about the value of the variable.
\begin{example}
The following example demonstrates that such variables come in handy
if we want to project out some variables.
\begin{lstlisting}
p(1,1,1).
p(2,1,2).
p(X) :- p(X,_A,_A).
\end{lstlisting}
Note that both occurrences of \code{\_A} are treated as distinct variables (we get both~\code{p(1)} and~\code{p(2)}).
\end{example}
\end{newstuff}

\subsubsection{Classical Negation}\label{subsec:gringo:negation}

In logic programs, connective \code{not} expresses default negation,
that is, a literal $\code{not}~A$ is assumed to hold unless~$A$ is derived.
In contrast, the classical (or strong) negation of some proposition
holds if the complement of the proposition is objectively derived~\cite{gellif91a}.
Classical negation, indicated by symbol ``\code{-},'' is permitted in front of atoms.
That is, if~$A$ is an atom, then~$\code{-}A$ is the complement of~$A$.
Semantically, $\code{-}A$ is simply a new atom,
with the additional condition that~$A$ and~$\code{-}A$ must not jointly hold.
A logic program~$\Pi$ containing complementary atoms is thus translated
to propositional theory
$\Phi[\Pi \cup \{\code{:-}~A\code{,}\code{-}A \mid A\in\ground{\mathcal{A}}\}]$.%
\footnote{Recall that $\ground{\mathcal{A}}$ consists of all variable-free atoms
built from predicates and functions in~$\Pi$.}
Observe that classical negation is merely a syntactic feature that can be
implemented via integrity constraints whose effect is to eliminate
any answer set candidate containing complementary atoms.

\begin{example}\label{ex:flies:neg}
The following logic program reformulates the one in Example~\ref{ex:flies}:
%
\lstinputlisting[nolol]{examples/bird.lp}
\lstinputlisting[firstnumber=3]{examples/flycn.lp}
%
Logically, classical negation is reflected by
(implicit) integrity constraints as follows:%
\marginlabel{By invoking\\
  \code{\mbox{~}gringo -t \textbackslash\\
        \mbox{~}examples/bird.lp \textbackslash\\
        \mbox{~}examples/flycn.lp}
the reader can observe that \gringo\ indeed produces the integrity constraint
in Line~7.}
%
\begin{lstlisting}[firstnumber=6]
:- flies(tux),    -flies(tux).
:- flies(tweety), -flies(tweety).
\end{lstlisting}
The additional integrity constraints do not yet change the semantics of the
program, which still has the answer sets already provided in Example~\ref{ex:flies:ground}
(of course, identifying
\code{\pred{neg\mus flies}(\const{tux})} with
\code{\pred{-flies}(\const{tux})} and
\code{\pred{neg\mus flies}(\const{tweety})} with
\code{\pred{-flies}(\const{tweety})}).
This situation changes if we add the following fact:
\begin{lstlisting}[firstnumber=8]
flies(tux).
\end{lstlisting}
While the program from Example~\ref{ex:flies} still admits two answer sets,
both containing 
\code{\pred{flies}(\const{tux})} and
\code{\pred{neg\mus flies}(\const{tux})},
there no longer is any answer set for our new program using classical negation.
In fact, answer set candidates that contain both
\code{\pred{flies}(\const{tux})} and
\code{\pred{-flies}(\const{tux})} violate the integrity constraint in Line~6.
\eexample
\end{example}


\subsubsection{Disjunction}\label{subsec:gringo:disjunction}

Disjunctive logic programs permit connective~``\code{|}'' between atoms in rule heads.
An additional case of translation~$\psi$ from Section~\ref{subsec:semantics}
reflects the semantics of disjunction:
\begin{itemize}
\item $\psi[G\code{|}H]=\psi[G]\vee\psi[H]$.
\end{itemize}
%
The concept of safe programs (cf.\ Section~\ref{subsec:safe}) directly applies
to disjunctive programs.
Furthermore, a (maximal) stratified subprogram (cf.\ Section~\ref{subsec:strat})
must not contain any rule with a (non-trivial) disjunctive head.

The following rule combines the ones in Line~3 and~4 from Example~\ref{ex:flies:neg}:
\begin{lstlisting}[numbers=none]
flies(X) | -flies(X) :- bird(X).
\end{lstlisting}
In general, the use of disjunction however increases
computational complexity~\cite{eitgot95a}.
This is why \clingo%
\footnote{Run as a monolithic system performing both grounding and solving.}
and solvers like 
\assat~\cite{linzha04a},
\clasp~\cite{gekanesc07b},
\nomorepp~\cite{angelinesc05c},
\smodels~\cite{siniso02a}, and
\smodelscc~\cite{warsch04a}
do not work on disjunctive programs.
Rather,
\claspD~\cite{drgegrkakoossc08a},
\cmodels~\cite{gilima06a,lierler05a}, or
\gnt~\cite{janisesiyo06a}
needs to be used for solving a disjunctive program.%
\footnote{System \dlv~\cite{dlv03a} also deals with disjunctive programs,
  but it uses a different syntax than presented here.}
We thus suggest to use ``choice constructs'' (cf.\ Section~\ref{subsec:gringo:aggregate})
instead of disjunction, unless the latter is required for complexity reasons
(see~\cite{eitpol06a} for an implementation methodology in disjunctive ASP).


\subsubsection{Built-In Arithmetic Functions}\label{subsec:gringo:arith}

\gringo\ and \clingo\ support a number of arithmetic functions that
are evaluated during grounding.
The following symbols are used for these functions:
\code{+} (addition),
\code{-} (subtraction),
\code{*} (multiplication),
\code{/} or \code{\#div} (integer division),
\code{\#mod} (modulo function),
\code{\#abs} (absolute value), and
\code{**} or \code{\#pow} (power function).
\code{\&} (bitwise AND).
\code{\^} (bitwise XOR).
\code{?} (bitwise OR).
\code{\~} (bitwise negate).

\begin{example}\label{ex:arith:fun}
The usage of arithmetic functions is illustrated by the logic program:%
\marginlabel{%
  The unique answer set of the program,
  obtained after evaluating all arithmetic functions,
  can be inspected by invoking:\\
  \code{\mbox{~}gringo -t \textbackslash\\\mbox{~}examples/arithf.lp}}
%
\lstinputlisting{examples/arithf.lp}
%
Note that variables~\var{L} and~\var{R} are instantiated to~\const{7} and~\const{2},
respectively, before arithmetic evaluations.
Consecutive and non-separative (e.g., before ``\code{(}'')
spaces can also be dropped,
while spaces around tokens \code{div} and \code{mod} are mandatory.
Furthermore, the argument of function \code{abs} must be enclosed in parentheses.
In Line~9, observe that there is a unary version of \code{-},
``\code{- \var{R}}'' standing for ``\code{0 - \var{R}}.''
The four bitwise functions apply to signed integers,
using the complement on two of a negative integer.
\eexample
\end{example}

\begin{newstuff}
It is important to note that variables
in the scope of an arithmetic function are not bound 
(in the sense of Section~\ref{subsec:safe}) by a corresponding (positive) atom.
Only edges for variables that are not in the scope of arithmetic functions
are added to the variable graph~$G=(V,E)$.
For instance, atom \code{\pred{p}(\var{X}+1,\var{X})} provides edge~$(\$, X)$,
while atom \code{\pred{p}(\var{X}+1,\var{Y})} provides edge~$(\$, Y)$,
but does not provide edge~$(\$,X)$ because it contains~\var{X}
only in the scope of an arithmetic function.

Note that previous version of \gringo\ also supported the arithmetic functions
\code{\textasciitilde} (bitwise complement),
\code{\&}  (bitwise AND),
\code{?}   (bitwise OR), and
\code{\^} (bitwise exclusive OR).
These have been removed because there are scarcely application applications using them.
Furthermore, with \gringo's new Lua support\comment{Reference to lua section}\ arbitrary arithmetic functions can be implemented.
\end{newstuff}

\subsubsection{Built-In Comparison Predicates}\label{subsec:gringo:comp}

The following built-in predicates permit term comparisons
within the bodies of rules and on the right-hand side of conditions
(cf.\ Section~\ref{subsec:gringo:condition}):
\code{==} (equal),
\code{!=} (not equal),
\code{<} (less than),
\code{<=} (less than or equal),
\code{>} (greater than),
\code{>=} (greater than or equal).

\begin{example}\label{ex:arith:pred}
The usage of comparison predicates is illustrated by the logic program:%
\marginlabel{%
  The unique answer set of the program is obtained via call:\\
  \code{\mbox{~}gringo -t \textbackslash\\\mbox{~}examples/arithc.lp}}
%
\lstinputlisting{examples/arithc.lp}
%
The last two lines hint at the fact that arithmetic functions are evaluated
before comparison predicates, so that the latter actually compare integers.

All comparison predicates can also be used with constants,
as in the next program:%
\marginlabel{%
  As above, invoking:\\
  \code{\mbox{~}gringo -t \textbackslash\\\mbox{~}examples/symbc.lp}
  yields the unique answer set of the program in terms of facts.}
%
\lstinputlisting{examples/symbc.lp}
%
Finally, note that~\code{==} and~\code{!=} also apply to
compound terms over functions with non-zero arity as well as to
mixed integer and non-integer arguments.%
\footnote{%
  Such functionalities are currently not supported for the other four
  comparison predicates,
  where (after instantiation and arithmetic evaluation)
  both arguments must be either integers or (symbolic) constants.
}
\eexample
\end{example}

\begin{newstuff}
Importantly, built-in comparison predicates do not provide edges to variable graphs (see Section~\ref{subsec:safe}).
\end{newstuff}

\subsubsection{Assignments}\label{subsec:gringo:assign}

Built-in predicate~\code{:=}%
\footnote{Note that there is also the built-in predicate~\code{=},
which, in contrast to~\code{:=}, only permits variables on its left hand side.
Apart from that both literals are semantically the same.
Predicate~\code{=} is mainly kept for compatibility with \lparse{} and older \gringo{} versions.} 
can be used in the body of a rule
(or on right-hand sides of conditions introduced in Section~\ref{subsec:gringo:condition})
to unify a term on its right-hand side to a term on its left-hand side.

%Note that the variable~$X$ in ``\code{\var{X} \!=\! }$t$'' is treated like a positive occurence if
%the variables on the right-hand side are bound positively minus cyclic assignments.
\begin{example}\label{ex:assign}
The next program demonstrates how terms can be assigned to variables:%
\marginlabel{%
  The unique answer set of the program is obtained via call:\\
  \code{\mbox{~}gringo -t \textbackslash\\\mbox{~}examples/assign.lp}}
%
\lstinputlisting{examples/assign.lp}
%
Line~3 contains four assignments, where the right-hand sides directly or indirectly
depend on~\var{X} and~\var{Y}.
These two variables are bound in Line~4 via atoms of predicate \pred{num}/$1$.
Also observe the different usage and role of built-in comparison predicate~\code{==}.
\eexample
\end{example}

\begin{newstuff}
For an assignemt literal~$A$, the variable graph (see Section~\ref{subsec:safe}) contains edges from 
the variables of the right hand sight to the variables of the left hand side of~$A$.
If a variable on the right hand side occurs in the scope of an arithmetic function,
then no edge is added.
If the right hand side does not contain variables then edges start with~$\$$.
For instance, literal \code{f(X,X+Y):=f(U,V)} provides the edges $(U,X)$ and $(V,X)$.
Intuitively, this means the assignment can be used to bind $X$ if $U$ and $V$ are bound by some other literal.
\end{newstuff}

\subsubsection{Intervals}\label{subsec:gringo:interval}

In Line~1 of Example~\ref{ex:assign},
there are five facts~\code{\pred{num}($k$)}
over consecutive integers~$k$.
For a more compact representation,
\gringo\ and \clingo\ support integer intervals of the form $i$\code{..}$j$,
where~$i$ and~$j$ are integers.
Such an interval represents each integer~$k$ such that $i\leq k\leq j$,
and intervals are expanded during grounding.

\begin{example}\label{ex:int}
The next program makes use of integer intervals:
%
\lstinputlisting{examples/int.lp}
%
The facts in Line~1 and~2 are expanded as follows:
%
\begin{lstlisting}[numbers=none]
num(1).    num(2).   num(3).   num(4).   num(5).
top5(5).  top5(6).  top5(7).  top5(8).  top5(9).
\end{lstlisting}
%
By instantiating~\var{X} to~\const{9},
the rule in Line~4 becomes:
%
\begin{lstlisting}[numbers=none]
top5num(1..5,5..9) :- num(5..9), top5(1..5), top(9).
\end{lstlisting}
%
It is expanded to the cross product
$(\code{\const{1}..\const{5}})\times
 (\code{\const{5}..\const{9}})\times
 (\code{\const{5}..\const{9}})\times
 (\code{\const{1}..\const{5}})$
of intervals:
%
\begin{lstlisting}[numbers=none,escapechar=@]
top5num(1,5) :- num(5), top5(1), top(9).
top5num(2,5) :- num(5), top5(1), top(9).
        @\rlap{\vdots}@
top5num(5,5) :- num(5), top5(1), top(9).
top5num(1,6) :- num(5), top5(1), top(9).
top5num(2,6) :- num(5), top5(1), top(9).
        @\rlap{\vdots}@  @\rlap{\vdots}@
top5num(5,9) :- num(5), top5(1), top(9).
top5num(1,5) :- num(6), top5(1), top(9).
top5num(2,5) :- num(6), top5(1), top(9).
        @\rlap{\vdots}@  @\rlap{\vdots}@          @\rlap{\vdots}@
top5num(5,9) :- num(9), top5(1), top(9).
top5num(1,5) :- num(5), top5(2), top(9).
top5num(2,5) :- num(5), top5(2), top(9).
        @\rlap{\vdots}@  @\rlap{\vdots}@          @\rlap{\vdots}@         @\rlap{\vdots}@
top5num(5,9) :- num(9), top5(4), top(9).
top5num(1,5) :- num(5), top5(5), top(9).
top5num(2,5) :- num(5), top5(5), top(9).
        @\rlap{\vdots}@  @\rlap{\vdots}@
top5num(5,9) :- num(5), top5(5), top(9).
top5num(1,5) :- num(6), top5(5), top(9).
top5num(2,5) :- num(6), top5(5), top(9).
        @\rlap{\vdots}@  @\rlap{\vdots}@          @\rlap{\vdots}@
top5num(5,9) :- num(9), top5(5), top(9).
\end{lstlisting}
%
Note that only the rules with
\code{\pred{num}(\const{5})} and
\code{\pred{top5}(\const{5})}
\marginlabel{%
  Again the unique answer set is obtained via call:\\
  \code{\mbox{~}gringo -t \textbackslash\\\mbox{~}examples/int.lp}}
%
in the body actually contribute to the unique answer set
of the above program by deriving all atoms \code{\pred{top5num}($m$,$n$)}
for $\const{1}\leq m\leq\const{5}$ and $\const{5}\leq n\leq\const{9}$.
\eexample
\end{example}

\begin{newstuff}
As with built-in arithmetic functions,
an integer interval mentioning some variable (like~\var{X} in Line~4 of Example~\ref{ex:int})
does not contribute edges in the variable graph, i.e., it will never bind variables.
\end{newstuff}

\subsubsection{Conditions}\label{subsec:gringo:condition}

Conditions allow for instantiating variables to
collections of terms within a single rule.
This is particularly useful for encoding conjunctions%
\footnote{Note that conditions also permit comparison predicates in front of the first colon.}
or disjunctions over
arbitrarily many ground atoms as well as for the compact representation of aggregates
(cf.\ Section~\ref{subsec:gringo:aggregate}).
The symbol~``\code{:}'' is used to formulate conditions.

\begin{example}\label{ex:cond}
The following program uses conditions in a rule body and in a rule head:
%
\lstinputlisting{examples/cond.lp}
%
We are particularly interested in the rules in Line~5 and~6,
instantiated as follows:%
%
\begin{lstlisting}[firstnumber=5]
meet :- available(jane), available(john).
on(mon) | on(tues) | on(wed) | on(thurs) | on(fri) :- meet.
\end{lstlisting}
%
The conjunction in Line~5 is obtained by replacing~\var{X} in
\marginlabel{%
  The reader can reproduce these ground rules by invoking:\\
  \code{\mbox{~}gringo -t \textbackslash\\\mbox{~}examples/cond.lp}}
\code{\pred{available}(\var{X})} with all ground terms~$t$ such that
\code{\pred{person}($t$)} holds, namely, $t=\const{jane}$ and $t=\const{john}$.
Furthermore, the condition in the head of the rule in Line~6 turns into
a disjunction over all ground instances of
\code{\pred{on}(\var{X})} where~\var{X} is substituted by some term~$t$
such that
\code{\pred{day}($t$)} holds.
That is, conditions in the body and in the head of a rule
are expanded to different basic language constructs.

Composite conditions can also be constructed via~``\code{:},''
as in the additional rules:
%
\lstinputlisting[firstnumber=7]{examples/twocond.lp}
%
Observe that we may use the same atom, viz., \code{\pred{day}(\var{X})},
both on the left-hand and on the right-hand side of~``\code{:}.''
Furthermore, negative literals like \code{not \pred{weekend}(\var{X})}
can occur on both sides of a condition.
Note that literals on the right-hand side of a condition are connected conjunctively, that is,
all of them must hold for ground instances of an atom in front
of the condition.
Thus, the instantiated rule in Line~8 looks as follows:
%
\begin{lstlisting}[firstnumber=8]
weekdays :- day(mon), day(tues), day(wed), day(thurs), day(fri).
\end{lstlisting}
%
The atoms in the body of this rule follow from facts, so that
the rule can be simplified to a fact \code{\pred{weekdays}.}
(as done by \gringo).
\eexample
\end{example}

There are three important issues about the correct usage of conditions.
First, all predicates of atoms on the right-hand side of a condition
must be either domain predicates (cf.\ Section~\ref{subsec:strat}) or built-in,
which is due to the fact that conditions are evaluated during grounding.%
\footnote{%
  The bodies of rules in a stratified subprogram (cf.\ Section~\ref{subsec:strat})
  may contain conditions.
  For a rule~$r$ and
  $L_0\code{:}L_1\code{:}\dots\code{:}L_n$
  in the body of~$r$,
  let $A_0\in\nbody{r}$ if $L_0=\code{not}~A_0$
  and $L_0\in\pbody{r}$ otherwise,
  and for $1\leq i\leq n$, 
  assume $A_i\in\nbody{r}$ for atom~$A_i$
  such that $L_i=A_i$ or $L_i=\code{not}~A_i$, respectively.
  }
Second, any variable occurring within a condition is considered as \emph{local},
that is, a condition cannot be used to bind variables outside itself.
In turn, variables outside conditions are \emph{global}, and each variable
within an atom in front of a condition must occur on the right-hand side or
be global.
Third, global variables take priority over local ones, that is,
they are instantiated first.
As a consequence, a local variable that also occurs globally is substituted by a term
before the ground instances of a condition are determined.
Hence, the names of local variables must be chosen with care,
making sure that they do not accidentally match the names of global variables.

\subsubsection{Pooling}\label{subsec:gringo:pool}

Symbol~``\code{;}'' allows for pooling alternative terms to be used as
argument within an atom, thus, specifying rules more compactly.
An atom written in the form \code{\pred{p}($\dots$,X;Y,$\dots$)}
abbreviates two options:
\code{\pred{p}($\dots$,X,$\dots$)} and \code{\pred{p}($\dots$,Y,$\dots$)}.
Pooled arguments in an atom of a rule body 
(or on the right-hand side of a condition) are expanded to a conjunction
of the options within the same body (or within the same condition),
while they are expanded to multiple rules 
(or multiple literals connected via~``\code{,}'')
when occurring in the head (or in front of a condition).

\begin{example}\label{ex:pool}
The following logic program makes use of pooling:
%
\lstinputlisting{examples/pool.lp}
%
Let us consider instantiations of the rule in Line~3 obtained with substitution 
$\{\var{A}\mapsto\const{a},\linebreak[1]\var{B}\mapsto\const{b},
   \var{M}\mapsto\const{1},\var{N}\mapsto\const{2}\}$.
Note that \pred{mix}/$2$ and \pred{-mix}/$2$ each admit four options,
corresponding to the cross product of $\{\const{a},\const{b}\}$ substituted
for~\var{A} and~\var{B}, respectively, together with $\{\const{1},\const{2}\}$
substituted for~\var{M} and~\var{N}.
While the instances obtained for \pred{mix}/$2$ give rise to four rules,
the instances for \pred{-mix}/$2$ jointly belong to the body.
The (repeated) body also contains two instances each of \pred{sym}/$1$ and of \pred{num}/$1$.
We thus get the rules:%
\marginlabel{%
  Simplified versions of these rules are produced via call:\\
  \code{\mbox{~}gringo -t \textbackslash\\\mbox{~}examples/pool.lp}}
%
\begin{lstlisting}[numbers=none]
mix(a,1) :- sym(a),sym(b), num(1),num(2), not -mix(1,a),
  not -mix(1,b), not -mix(2,a), not -mix(2,b).
mix(a,2) :- sym(a),sym(b), num(1),num(2), not -mix(1,a),
  not -mix(1,b), not -mix(2,a), not -mix(2,b).
mix(b,1) :- sym(a),sym(b), num(1),num(2), not -mix(1,a),
  not -mix(1,b), not -mix(2,a), not -mix(2,b).
mix(b,2) :- sym(a),sym(b), num(1),num(2), not -mix(1,a),
  not -mix(1,b), not -mix(2,a), not -mix(2,b).
\end{lstlisting}
%
Finally, we note that pooling is possible in every term.
Although, it is most useful in predicates.
\eexample
\end{example}

\begin{newstuff}
There is a second pooling operator \code{;;} that is only permitted in predicates.
It is used to seperate arguments of predicates,
e.g., \code{p(a,b;;c,d)} expands to \code{p(a,b), p(c,d)}.
Note that this operator corresponds to \lparse's~\code{;} operator.
\end{newstuff}

\subsubsection{Aggregates}\label{subsec:gringo:aggregate}
\com{Rewrote much here}

An aggregate is an operation on a multiset of weighted literals that evaluates to a value,
depending on the type of aggregate this might be integers but, 
e.g., \const{\#min} and \const{\#max} aggregates also permit function symbols.
In combination with comparisons, we can extract a truth value from
an aggregate's evaluation, thus, obtaining an aggregate atom.
Customizing the notation in~\cite{ferraris05a},
we consider ground \emph{aggregate atoms} of the form:
%
\begin{equation}\label{eq:aggregate:atom}
  l~ 
  \mathrm{op} 
  \code{[}
  L_1\code{=}w_1\code{,}
  \dots\code{,}
  L_n\code{=}w_n
  \code{]}
  u
\end{equation}
%
where~$l$ and~$u$ are values each of which can possibly be omitted,
$\mathrm{op}$ is a function from multisets of values to a value,
each $L_i$ is a literal of the form~$A$ or $\code{not}~A$
for some atom~$A\in\ground{\mathcal{A}}$,
and each~$w_i$ is a value.
The intuitive reading of an aggregate atom is that the value returned by~$\mathrm{op}$
applied to all weights~$w_i$ such that~$L_i$ holds should be in-between
lower bound~$l$ and upper bound~$u$ (inclusively).
If~$l$ or~$u$ is omitted,
it means that the aggregate's value is not constrained from below or above,
respectively.

For an aggregate atom~$\mathit{Agg}$ as in Equation(\ref{eq:aggregate:atom}),
we let $\atom{\mathit{Agg}}=\{L_i\in\nolinebreak\ground{\mathcal{A}}\mid 1\leq i\leq n\}$
denote the set of atoms occurring positively in~$\mathit{Agg}$.
Currently, \gringo\ and \clingo\ support aggregate operations
\const{\#sum}, \const{\#min}, and \const{\#max}, \const{\#avg}, \const{\#even}, and \const{\#odd}.%
\footnote{%
  The additional operation \const{count} is a shorthand for a \const{sum}
  aggregate in which all weights are~\const{1}.}
We can now extend translations~$\phi$ and~$\psi$ from Section~\ref{subsec:semantics}
to aggregate atoms, 
using the approach in~\cite{ferraris05a} but
applying the ``choice semantics'' for aggregates
in heads of rules~\cite{ferlif05a,siniso02a}:%
 
\newcommand\shiftleft{\kern\leftshift&\kern-\leftshift}

\newlength{\leftshift} 
\setlength{\leftshift} {2em}
%\settowidth{\leftshift}{$\phi^{+}[ l~ \const{\#sum} \code{[} L_1\code{=}w_1\code{,} \dots\code{,} L_n\code{=}w_n \code{]} ]$}

\begin{align}
  % l op u (head) %
  \shiftleft
  \psi[ \lhd~ \mathrm{op} \code{[} L_1\code{=}w_1\code{,} \dots\code{,} L_n\code{=}w_n \code{]}\rhd ] 
  \label{eq:aggr_head} \\
   &= \phi[ \triangleleft~ \mathrm{op} \code{[} L_1\code{=}w_1\code{,} \dots\code{,} L_n\code{=}w_n \code{]}\triangleright ] \notag \\
   &\wedge \bigl(\bigwedge\nolimits_{A\in atom(\triangleleft~ \mathrm{op} \code{[} L_1\code{=}w_1\code{,} \dots\code{,} L_n\code{=}w_n \code{]}\triangleright)}(A\vee\neg A)\bigr) \notag \\
  % l op u %
  \shiftleft
  \phi[ l~ \mathrm{op} \code{[} L_1\code{=}w_1\code{,} \dots\code{,} L_n\code{=}w_n \code{]} u ]
  \label{eq:aggr_lu} \\
  &= \phi[ l~ \mathrm{op} \code{[} L_1\code{=}w_1\code{,} \dots\code{,} L_n\code{=}w_n \code{]} ] \wedge{} 
    \phi[ \mathrm{op} \code{[} L_1\code{=}w_1\code{,} \dots\code{,} L_n\code{=}w_n \code{]} u ] \notag \\
  % op %
  \shiftleft
  \phi[ \mathrm{op} \code{[} L_1\code{=}w_1\code{,} \dots\code{,} L_n\code{=}w_n \code{]} ] = \top
  \label{eq:aggr}
\end{align}
Translation~$\psi$ of an aggregate atom in the head of a rule consists of two parts:
first, $\phi$ is used as defined in the following
(in Equation (\ref{eq:aggr_head}), $\lhd$ and~$\rhd$ are used as ``wildcards'' for provided or omitted 
 lower and upper bounds),
second, a disjunction $(A\vee\neg A)$ is added for each atom~$A$ occurring positively.
Even though such disjunctions are tautological,
they are important under answer set semantics (cf.\ Section~\ref{subsec:semantics}),
as they permit deriving an arbitrary subset of atoms from an aggregate in the head
(which explains the name  ``choice semantics'').
Equation \ref{eq:aggr_lu} splits up an aggregate with lower and an upper into the conjunction of two
conjunctively connected parts.
Note that this split agrees with the semtaics in \cite{ferraris05a} independend of the aggregate $\mathrm{op}$ used.
Finally, Equation (\ref{eq:aggr}) reflects that an unconstrained aggregate atom always holds.

\paragraph{Sum Aggregates}
First, we show how monotone aggregates can be translated into a nested expression.
Using this translation we show how to transfom antimonotone sum aggregates and 
also give a translation that treats negative weights.
\footnote{%
  The translation~$\phi$ we provide for \const{sum} aggregates
  assumes that there are no negative weights for literals,
  while the more sophisticated translation in~\cite{ferraris05a} also
  captures negative weights.
  All current solvers work with positive weights only, and thus
  \gringo\ and \lparse\ use a compilation technique~\cite{siniso02a}
  to eliminate negative weights, which are not permitted in
  \lparse's output format~\cite{lparseManual}.
  As this technique may lead to counterintuitive results~\cite{ferraris05a},
  we strongly recommend not to use negative weights within \const{sum} aggregates.}
\begin{align}
  % l sum+ %
  \shiftleft
  \phi[ l~ \const{\#sum}^{+} \code{[} L_1\code{=}w_1\code{,} \dots\code{,} L_n\code{=}w_n \code{]} ] 
  \label{eq:sum_plus}\\
  &= \bigvee\nolimits_{\{i_1,\dots,i_j\}\subseteq\{1,\dots,n\},l\leq w_{i_1}+\dots+w_{i_j}}
   (\phi[L_{i_1}]\wedge\dots\wedge \phi[L_{i_j}]) \notag 
\end{align}
Equation (\ref{eq:sum_plus}) translates a monotone aggregate~$\const{\#sum}^{+}$
with a lower bound and only positive weights into a nested expressions.

\begin{align}
  % l sum %
  \shiftleft
  \phi[l~\const{\#sum}~\code{[} L_1\code{=}w_1\code{,} \dots\code{,} L_n\code{=}w_n \code{]} ]
  \label{eq:sum_neg} \\
  &= \phi [ l + \phi^{+}(w_1) + \dots + \phi^{+}(w_n) ~\const{\#sum}^{+}
    \code{[} \phi^{+}(L_1\code{=}w_1)\code{,} \dots\code{,} \phi^{+}(L_n\code{=}w_n) \code{]} ] \notag \\
  % weight lit %
  \shiftleft
  \phi^{+}(L=w) = 
    \left\{\begin{aligned}
      L=w & \mbox{ if } w \geq 0 \\
      \overline{L}=-w &  \mbox{ otherwise}\\
    \end{aligned}\right. \notag \\
  % weight %
  \shiftleft
  \phi^{+}(w) =-\min\{0,w\} \notag
\end{align}
Equation (\ref{eq:sum_neg}) removes all negative weights from a sum aggregate with a lower bound 
and then applies the transformation described above.
Whenever there is a negative weight, the sign of both literal and weight is inverted and the weight itself is subtracted from the bound.
Note that this transformation produces counter-intuitive results as soon as the aggregate is used recursively.
Nevertheless, there are scenarios where negative weights might be helpfull.
As long as the aggregate is stratified, the transformation corresponds with the intuitive meaning, e.g.,
you can use negative weights to compare two sets $A=\{a,b,c\}$ and $B=\{d,e\}$:
\begin{lstlisting}[numbers=none]
subseteq(A,B) :- #sum [ a=1,b=1,c=1,d=-1,e=-1 ] ] 0.
\end{lstlisting}

\begin{align}
  % sum u %
  \phi[ \const{\#sum} \code{[} L_1\code{=}w_1\code{,} \dots\code{,} L_n\code{=}w_n \code{]}~u ] 
  &= \neg \phi[  u+1~\const{\#sum} \code{[} L_1\code{=}w_1\code{,} \dots\code{,} L_n\code{=}w_n \code{]}]
  \label{eq:sum_u}
\end{align}
Finally, Equation (\ref{eq:sum_u}) translates sum aggregates with an upper bound using the
translation for sum aggregates with lower bounds described above.
Note that again negative literals might cause problems.
Even in the case that all weights are negative (and as such the sum aggregate itself is monotone),
the above translation is used, which might lead to counter-intuitive results.

\paragraph{Max and Min Aggregates} 
We define the sematics of max aggregates using the transformation for sum aggregates described above.
First, we translate max aggregates with upper and lower bound, respectively, into corresponding sum aggregates.
Then we give the semantics of min aggregates using the translation for max aggregates.
Note that in constrast to sum aggregates, max aggregates allow for using non-integer weights as long as there
is some strict total ordering defined on them.
\begin{align}
  % l max
  \shiftleft
  \phi[l~\const{\#max} \code{[} L_1\code{=}w_1\code{,} \dots\code{,} L_n\code{=}w_n \code{]} ]
  \label{eq:max_l}\\
  &= \phi[1~\const{\#sum} \code{[} L_1\code{=}\mathrm{ge}(w_1, l)\code{,} \dots\code{,} L_n\code{=}\mathrm{ge}(w_1, l) \code{]} ]  \notag \\
  % max(w,l)
  \shiftleft
  \mathrm{ge}(a, b) = \left\{\!\begin{aligned}
       1 &\mbox{ if } a \geq b \\
       0 &\mbox{ otherwise}
     \end{aligned}\right. \notag
\end{align}
A (monotone) max aggregate with a lower bound is translated into a sum by setting all weights 
that are greater equal to the lower bound to one and requiring that the sum of weights is at least one.
This transformation preserves the monotonicity of the aggregate and holds independendly of the type of weights used.

\begin{align}
  % max u %
  \shiftleft
  \phi[\const{\#max} \code{[} L_1\code{=}w_1\code{,} \dots\code{,} L_n\code{=}w_n \code{]} u]
  \label{eq:max_u} \\
  &= \phi[\const{\#sum} \code{[} L_1\code{=}\mathrm{ge}(w_1, l)\code{,} \dots\code{,} L_n\code{=}\mathrm{ge}(w_1, l) \code{]} 0]  \notag \\
  \shiftleft
  \mathrm{gt}(a, b) = \left\{\!\begin{aligned}
       1 &\mbox{ if } a > b \\
       0 &\mbox{ otherwise}
     \end{aligned}\right. \notag
\end{align}
Similarly to Equation~(\ref{eq:max_l}), an (antimonotone) max aggregate with an upper bound can be translated into a sum with an upper bound.
Again the monotonicity of the aggregate is preserved.

\begin{align}
  % min u %
  \shiftleft
  \phi[ \const{\#min} \code{[} L_1\code{=}w_1\code{,} \dots\code{,} L_n\code{=}w_n \code{]}~u ] 
  \label{eq:min_u}  \\
  &= \phi[1~\const{\#sum} \code{[} L_1\code{=}\mathrm{ge}(l,w_1)\code{,} \dots\code{,} L_n\code{=}\mathrm{ge}(l,w_1) \code{]}]
  \notag \\
  % l min %
  \shiftleft
  \phi[ l~ \const{\#min} \code{[} L_1\code{=}w_1\code{,} \dots\code{,} L_n\code{=}w_n \code{]} ]
  \label{eq:min_l} \\
  &= \phi[\const{\#sum} \code{[} L_1\code{=}\mathrm{gt}(l,w_1)\code{,} \dots\code{,} L_n\code{=}\mathrm{gt}(l,w_1) \code{]} 0]
  \notag
\end{align}
Analogous to the definition of max aggregates, 
Equations (\ref{eq:min_l}) and (\ref{eq:min_u}) define the semantics of min aggregates.

\paragraph{Average Aggregates}
Avergate can be directly translated into sum aggregates by modifying the weights of its literals.
To see how the translation works consider:
\begin{align*}
\mathrm{avg}\code{[}a_1,\dots,a_n\code{]} &\prec b \\
a_1+\dots+a_n               &\prec n \cdot b \\
a_1 - b + \dots + a_n - b   &\prec 0
\end{align*}
where $\prec$ is some relation.
To avoid problems with empty average aggregates we define that $\mathrm{avg}\code{[} \code{]} \prec b$ iff $0 \prec 0$.
Thus the translation can turn an average aggregate into a sum by subtracting the bound from all its weights.
A strong point of this translation is that it preserves the monotonicity of the average aggregate.
This translation is applied in Equations (\ref{eq:avg_l}) and (\ref{eq:avg_u}).

\begin{align}
  % l avg %
  \shiftleft
  \phi[ l~\const{\#avg} \code{[} L_1\code{=}w_1\code{,} \dots\code{,} L_n\code{=}w_n \code{]} ] 
  \label{eq:avg_l} \\
  &= \phi[ \const{\#sum} \code{[} L_1\code{=}w_1-l \code{,} \dots\code{,} L_n\code{=}w_n-l \code{]}~0 ] 
  \notag \\
  % avg u %
  \shiftleft
  \phi[ \const{\#avg} \code{[} L_1\code{=}w_1\code{,} \dots\code{,} L_n\code{=}w_n \code{]}~u ] = 
  \label{eq:avg_u} \\
  &= \phi[ \const{\#sum} \code{[} L_1\code{=}w_1-u \code{,} \dots\code{,} L_n\code{=}w_n-u \code{]}~0 ]
  \notag
\end{align}

\paragraph{Parity Aggregates}
We also support the two parity aggregates $\code{\#even}$ and $\code{\#odd}$.
Intuitively, an even or odd aggregate is true if the sum of its weights modulo two is one or zero, respectively.
Note that we do not write bounds
Even and odd aggregates are transformed into nested expressions:
\begin{align}
  % odd %
  \shiftleft
  \phi[ \const{\#odd} \code{[} L_1\code{=}w_1\code{,} \dots\code{,} L_n\code{=}w_n \code{]} ] \\
  &=\left\{\begin{aligned}
    \bot & \mbox{ if } n = 0 \\
    \phi[ \const{\#odd} \code{[} L_2\code{=}w_2-u \code{,} \dots\code{,} L_n\code{=}w_n-u \code{]}~0 ] & \mbox{ if } 2 | w_1  \\
    \neg L_1 \wedge \phi[ \const{\#odd} \code{[} L_2\code{=}w_2 \code{,} \dots\code{,} L_n\code{=}w_n \code{]}~0 ] & \\
    \vee L_1 \wedge \neg \phi[ \const{\#odd} \code{[} L_2\code{=}w_2 \code{,} \dots\code{,} L_n\code{=}w_n \code{]}~0 ] & \mbox{ otherwise}
  \end{aligned}\right. \notag
\end{align}

\begin{align}
  % even %
  \shiftleft
  \phi[ \const{\#even} \code{[} L_1\code{=}w_1\code{,} \dots\code{,} L_n\code{=}w_n \code{]} ] \\
  &=\left\{\begin{aligned}
    \top & \mbox{ if } n = 0 \\
    \phi[ \const{\#even} \code{[} L_2\code{=}w_2-u \code{,} \dots\code{,} L_n\code{=}w_n-u \code{]}~0 ] & \mbox{ if } 2 | w_1  \\
    \neg L_1 \wedge \phi[ \const{\#even} \code{[} L_2\code{=}w_2 \code{,} \dots\code{,} L_n\code{=}w_n \code{]}~0 ] & \\
    \vee L_1 \wedge \neg \phi[ \const{\#even} \code{[} L_2\code{=}w_2 \code{,} \dots\code{,} L_n\code{=}w_n \code{]}~0 ] & \mbox{ otherwise}
  \end{aligned}\right. \notag
\end{align}

Item~3--8 specify these parts individually for~\const{sum},~\const{max}, and~\const{min}
aggregates, respectively.
Provided that only non-negative weights are used within \const{sum}s,
all three aggregates are subject to one comparison that is monotone,
meaning that by default it evaluates to \textit{false} so that truth must be established.
For \const{sum} and \const{max} aggregates, these monotone comparisons are related
to lower bounds (Item~3 and~5),
while an upper bound behaves monotonically for \const{min} (Item~8).
In turn, the opposite comparison is antimonotone, that is,
by default it evaluates to \textit{true} unless too many or improper literals hold.
This is the case for comparisons to upper bounds with aggregates
\const{sum} and \const{max} (Item~4 and~6), while it concerns 
the lower bound for \const{min} (Item~7).
% <cut here>
Beyond this special treatment in rule heads,
the truth values of aggregate atoms result from their literals as
may be intuitively expected,
and Item~1--8 basically do nothing but formalizing the standard meanings
of aggregate operations.
Note that the above translations merely define the semantics of aggregates,
which does not imply that solvers unfold them in this way.
In fact, \clasp\ natively supports aggregates without translating them.

As regards syntactic representation,
weight~\const{1} is considered a default,
so that $L_i\code{=}\const{1}$ can simply be written as $L_i$.
For instance, the following (multi)sets of (weighted) literals are the same
when combined with any kind of aggregate operation and bounds:
\begin{lstlisting}[numbers=none,escapechar=@]
[a=1, not b=1, c=2]  @\textnormal{and}@
[a,   not b,   c=2]@\textnormal{.}@
\end{lstlisting}
%
Furthermore,
keyword \const{sum} may be omitted, which in a sense makes \const{sum} the default
aggregate operation.
In fact, the following aggregate atoms are synonyms:
\begin{lstlisting}[numbers=none,escapechar=@]
2 sum [a, not b, c=2] 3  @\textnormal{and}@
2     [a, not b, c=2] 3@\textnormal{.}@
\end{lstlisting}
By omitting keyword \const{sum}, we obtain the same notation as the one of 
so-called ``weight constraints''~\cite{siniso02a,lparseManual}, which are actually aggregate atoms
whose operation is addition.

It is important to note that the (weighted) literals within an aggregate
belong to a multiset.
In particular, if there are multiple occurrences
$L\code{=}w_1,\dots,L\code{=}w_k$
of a literal~$L$, in combination with \const{min} and \const{max},
it is not the same like having $L\code{=}w_1+\dots+w_k$.
To see this, note that the program consisting of the facts:
\begin{lstlisting}[numbers=none]
2 max [a=2].  2 min [a=2].
\end{lstlisting}
has $\{\code{a}\}$ as its unique answer set, while there is no answer set for:
\begin{lstlisting}[numbers=none]
2 max [a,a].  2 min [a,a].
\end{lstlisting}

If literals ought not to be repeated,
we can use \const{count} instead of \const{sum}.
Syntactically, \const{count} requires curly instead of square
brackets, and there must not be any weights within a \const{count} aggregate.
Regarding semantics,
% \begin{equation*}%\label{eq:aggregate:atom}
$(
  l~ 
  \const{count} 
  \code{\char`\{}
  L_1\code{,}
  \dots\code{,}
  L_n
  \code{\char`\}}
  u
)$ 
%\end{equation*}
reduces to
$(
  l~ 
  \const{sum} 
  \code{[}
  L_1\code{=}\const{1}\code{,}
  \dots\code{,}
  L_m\code{=}\const{1}
  \code{]}
  u
)$,
where $\{L_1,\dots,L_m\}=\{L_i \mid 1\leq i\leq n\}$ is obtained by dropping
repeated literals.
Of course, the use of~$l$ and~$u$ is optional also with \const{count}.
As an example, note that the next aggregate atoms express the same:
\begin{lstlisting}[numbers=none,escapechar=@]
1 sum   [a=1, not b=1]     1  @\textnormal{and}@ 
1 count {a,a, not b,not b} 1@\textnormal{.}@  
\end{lstlisting}
Keyword \const{count} can be omitted (like \const{sum}),
so that the following are synonyms:
\begin{lstlisting}[numbers=none,escapechar=@]
1 count {a, not b} 1  @\textnormal{and}@ 
1       {a, not b} 1@\textnormal{.}@  
\end{lstlisting}
The last notation is similar to the one of so-called
``cardinality constraints''~\cite{siniso02a,lparseManual},
which are aggregate atoms using counting as their operation.

After considering the syntax and semantics of ground aggregate atoms,
we now turn our attention to non-ground aggregates.
Regarding contained variables, an atom occurring in an aggregate behaves
similar to an atom on the left-hand side of a condition
(cf.\ Section~\ref{subsec:gringo:condition}).
That is, any variable occurring within an aggregate is a priori local,
and it must be bound via a variable of the same name that is global
or that occurs on the right-hand side of a condition
(with the atom containing the variable in front).
As with local variables of conditions, global variables take priority
during grounding, so that the names of local variables must be chosen
with care to avoid accidental clashes.
Beyond conditions (which are more or less the natural construct to use
for instantiating variables within an aggregate),
classical negation (cf.\ Section~\ref{subsec:gringo:negation}),
built-in arithmetic functions (cf.\ Section~\ref{subsec:gringo:arith}),
intervals (cf.\ Section~\ref{subsec:gringo:interval}), and 
pooling (cf.\ Section~\ref{subsec:gringo:pool})
can be incorporated as usual within aggregates,
where intervals and pooling are expanded locally.%
\footnote{%
  Assignments (cf.\ Section~\ref{subsec:gringo:assign}) and (currently)
  also built-in comparison predicates (cf.\ Section~\ref{subsec:gringo:comp})
  are permitted on the right-hand sides of conditions only.}
That is, an interval gives rise to multiple literals connected via~``\code{,}''
within the same aggregate.
The same applies to pooling in front of a condition,
while it turns into a composite condition chained by~``\code{:}'' on the right-hand side.
The notions of level-restrictedness and stratification
(cf.\ Section~\ref{subsec:lambda} and~\ref{subsec:strat}) are extended
to programs with aggregates by
considering the predicates of all atoms (except for those on the right-hand side of a condition)
in an aggregate atom occurring as the head of a rule~$r$,%
\footnote{%
  The rules in a (maximal) stratified subprogram (cf.\ Section~\ref{subsec:strat})
  cannot have aggregate atoms in the head.}
and by assuming $A\in\nbody{r}$ for all atoms~$A$ of an aggregate in the body of~$r$.
Finally, note that aggregates without bounds are also permitted on the
right-hand sides of assignments, but using this feature is only recommended
for aggregates whose atoms belong to domain predicates because space blow-up
can become a bottleneck otherwise.
The following example, making exhaustive use of aggregates,
nonetheless demonstrates this and other features.

\begin{example}\label{ex:aggr}
Consider a situation where an informatics student
wants to enroll for a number of courses at the beginning of a new term.
In the university calendar, eight courses are found eligible,
and they are represented by the following facts:
%
\lstinputlisting[xrightmargin=-20pt,lastline=8]{examples/aggr.lp}
%
In an instance of \const{course}/$3$,
the first argument is a number identifying one of the eight courses,
and the third argument provides the course's contact hours per week.
The second argument stands for a subject area:
\const{1} corresponding to ``theoretical informatics,''
\const{2}               to ``practical informatics,''
\const{3}               to ``technical informatics,''
and~\const{4}               to ``applied informatics.''
For instance, atom \code{\const{course}(\const{1},\const{2},\const{5})}
expresses that course~\const{1} accounts for~\const{5} contact hours per week
that may be credited to subject area~\const{2} (``practical informatics'').
Observe that a single course is usually eligible for multiple
subject areas.

After specifying the above facts,
the student starts to provide personal constraints on the courses to enroll.
The first condition is that~\const{3} to~\const{6} courses should be enrolled:
%
\lstinputlisting[nolol,firstline=11,lastline=11,firstnumber=9]{examples/aggr.lp}
%
Instantiating the above \const{count} aggregate
yields the following ground rule:%
\marginlabel{%
  The full ground program is obtained by invoking:\\
  \code{\mbox{~}gringo -t \textbackslash\\\mbox{~}examples/aggr.lp}}
%
\begin{lstlisting}[firstnumber=9,stepnumber=9]
3 { enroll(1), enroll(2), enroll(3), enroll(4), 
    enroll(5), enroll(6), enroll(7), enroll(8) } 6.
\end{lstlisting}
%
Observe that an instance of atom~\code{\const{enroll}(\var{C})} is included for each
instantiation of~\var{C} such that \code{\const{course}(\var{C},\var{S},\var{H})}
holds for some values of~\var{S} and~\var{H}.
Duplicates resulting from distinct values for~\var{S} are removed, thus,
obtaining the above set of ground atoms.

The next constraints of the student regard the subject areas of enrolled courses:
%
\lstinputlisting[nolol,firstline=13,lastline=15,firstnumber=10]{examples/aggr.lp}
%
Each of the three integrity constraints above contains a \const{sum} aggregate,
using default weight~\const{1} for literals.
Recalling that \const{sum} aggregates operate on multisets,
duplicates are not removed.
Thus, the integrity constraint in Line~10 is instantiated as follows:
%
\begin{lstlisting}[firstnumber=10,stepnumber=10]
:- [ enroll(1) = 1, enroll(1) = 1, 
     enroll(2) = 1, enroll(2) = 1,
     enroll(3) = 1, enroll(3) = 1,
     enroll(4) = 1, enroll(4) = 1, enroll(4) = 1,
     enroll(5) = 1, enroll(5) = 1,
     enroll(6) = 1, enroll(6) = 1,
     enroll(7) = 1, enroll(7) = 1, enroll(7) = 1,
     enroll(8) = 1, enroll(8) = 1 ] 10.
\end{lstlisting}
%
Note that courses~\const{4} and~\const{7} count three times because
they are eligible for three subject areas, viz., there are three
distinct instantiations for~\var{S} in
\code{\const{course}(\var{4},\var{S},\const{3})} and
\code{\const{course}(\var{7},\var{S},\const{4})}, respectively.
Comparing the above ground instance,
the meaning of the integrity constraint in Line~10 is that the 
number of eligible subject areas over all enrolled courses
must be more than~\const{10}.
Similarly, the integrity constraint in Line~11 expresses the requirement
that at most one course of subject area~\const{2} (``practical informatics'')
is not enrolled,
while Line~12 stipulates that the enrolled courses
amount to less than six nominations of
subject area~\const{3} (``technical informatics'') 
or~\const{4} (``applied informatics'').
Also note that, given the facts in Line~1--8,
we could equivalently have used \const{count} rather than \const{sum} in Line~11,
but not in Line~10 and~12.

The remaining constraints of the student deal with contact hours.
To express them, we first introduce an auxiliary rule and a fact:
%
\lstinputlisting[nolol,firstline=17,lastline=18,firstnumber=13]{examples/aggr.lp}
%
The rule in Line~13 projects instances of \pred{course}/$3$ to
\pred{hours}/$2$, thereby, dropping courses' subject areas.
This is used to not consider the same course multiple times within the following
integrity constraints:
%
\lstinputlisting[nolol,firstline=20,lastline=22,firstnumber=15]{examples/aggr.lp}
%
As Line~15 shows,
we may use default negation via ``\code{not}'' in front of aggregate atoms,
and bounds may be specified in terms of variables.
In fact, by instantiating~\var{M} to~\const{20},
we obtain the following ground instance of the integrity constraint in Line~15:
%
\begin{lstlisting}[firstnumber=15,stepnumber=15]
:- not 18 [ enroll(1) = 5, enroll(2) = 4, 
            enroll(3) = 6, enroll(4) = 3,
            enroll(5) = 4, enroll(6) = 2,
            enroll(7) = 4, enroll(8) = 5 ] 20.
\end{lstlisting}
%
The above integrity constraint states that the \const{sum} of contact hours per week
must lie in-between~\const{18} and~\const{20}.
Note that the \const{min} and \const{max} aggregates in Line~16 and~17, respectively,
work on the same (multi)set of weighted literals as in Line~15.
While the integrity constraint in Line~16 stipulates that any course to enroll
must include more than~\const{2} contact hours,
the one in Line~17 prohibits enrolling for courses of~\const{6} or more contact hours.
Of course, the last two requirements could also be formulated as follows:
%
\begin{lstlisting}[firstnumber=16]
:- enroll(C), hours(C,H), H <= 2.
:- enroll(C), hours(C,H), H >= 6.
\end{lstlisting}

Finally, the following rules illustrate the use of aggregates within assignments:
%
\lstinputlisting[nolol,firstline=24,lastline=25,firstnumber=18]{examples/aggr.lp}
%
Note that the above aggregates have already been used in Line~9 and~15, respectively,
where keywords \const{count} and \const{sum} have been omitted for convenience.
These keywords can be dropped here too, and we merely include them to show the
more verbose notations of \const{count} and \const{sum} aggregates.
However, the usage of aggregates in the last two lines is different from before,
as they now serve to assign an integer to a variable~\var{N}.
In this context, bounds are not permitted, and so none are provided in Line~18 and~19.
The effect of these two lines is that the student can read off the number of
courses to enroll and the amount of contact hours per week from instances of
\pred{courses}/$1$ and \pred{hours}/$1$ belonging to an answer set.%
\marginlabel{%
  To compute the unique answer set of the program, invoke:\\
  \code{\mbox{~}gringo \textbackslash\\
        \mbox{~}examples/aggr.lp | \textbackslash\\
        \mbox{~}clasp -n 0}\\
  or alternatively:\\
  \code{\mbox{~}clingo -n 0 \textbackslash\\
        \mbox{~}examples/aggr.lp}}
%
In fact, running \clasp\ shows the student that a unique
collection of~\const{5} courses to enroll satisfies all requirements:
the courses~\const{1}, \const{2}, \const{4}, \const{5}, and~\const{7},
amounting to~\const{20} contact hours per week.

Although the above program does not reflect this possibility,
it should be noted that (as has been mentioned in Section~\ref{subsec:gringo:condition})
multiple literals may be connected via ``\code{:}'' in order to construct
composite conditions within an aggregate.
As before, the predicates of atoms on the right-hand side of such conditions
must be either domain predicates or built-in.
Furthermore, the usage of non-domain predicates within an aggregate
on the right-hand side of an assignment (like \pred{enroll}/$1$ in Line~18 and~19 above)
is not recommended in general because the space blow-up may be significant.
\eexample
\end{example}

\begin{newstuff}
	\begin{itemize}
		\item there is also $\#$avg, $\#$times, $\#$even, $\#$odd
		\item bingo will support everything but times
		\item heads of stratified aggregates can bind their variables
		\item a \# has to be put in front of every aggregate $\Rightarrow$ adjust examples
	\end{itemize}
\end{newstuff}


\subsubsection{Optimization}\label{subsec:gringo:optimize}

Optimization statements extend the basic question of
whether a set of atoms is an answer set to
whether it is an optimal answer set.
To support this reasoning mode, \gringo\ and \clingo\ adopt the
optimization statements of \lparse~\cite{lparseManual},
indicated via keywords \const{maximize} and \const{minimize}.
Syntactically, a \const{maximize} or \const{minimize} statement
is similar to a fact whose head is a \const{count} or a \const{sum}
aggregate (without bounds),
viz.,
``$\textrm{opt}\code{\char`\{}\dots\code{\char`\}}\code{.}$'' or
``$\textrm{opt}\code{[}\dots\code{]}\code{.}$'' where
$\textrm{opt}\in\{\const{maximize},\linebreak[1]\const{minimize}\}$.
As an optimization statement does not admit a body, any (local) variable
in it must also occur in an atom (over a domain or built-in predicate)
on the right-hand side of a condition (cf.\ Section~\ref{subsec:gringo:condition})
within the optimization statement.
In multiset notation (square brackets),
weights may be provided as with \const{sum} aggregates. %,
% where~\const{1} is used as default for omitted weights.
In set notation (curly brackets),
duplicates of literals are removed
as with \const{count} aggregates.

The semantics of an optimization statement is intuitive:
an answer set is \emph{optimal} if the sum of weights
(using~\const{1} for unsupplied weights)
of literals that hold is maximal or minimal, as required by the statement,
among all answer sets of the given program.
This definition is sufficient if a single optimization statement is specified
along with a logic program.
Unfortunately, the syntax used by \lparse\ and adopted by \gringo\ and \clingo\
does not provide any means to explicitly declare priorities among multiple
optimization statements.
A technique how to compile multiple optimization statements into a single one
is discussed in~\cite{siniso02a},
but it can lead to large integers
(probably rather unnatural for a user to assign).
Thus, rather than restricting to a single optimization statement,
a sequence of them is permitted,
where a statement provided later on takes priority over previous ones.
That is, the last optimization statement in program is the most significant one,
and previous ones are only considered if answer sets agree on the sum of 
weights of its literals.
In this way, multiple optimization statements are arranged into a total order,
which is the inverse order of occurrence.
In contrast to the declarative answer set semantics, the semantics of
optimization statements is thus order-dependent,
requiring special care when providing more than one of them.

\begin{example}\label{ex:opt}
To illustrate optimization, we consider a hotel booking situation
where we want to choose one among five available hotels.
The hotels are identified via numbers assigned in descending order of stars.
Of course, the more stars a hotel has the more it costs per night.
As an ancillary information, we know that hotel~\const{4} is located
on a main street, which is why we expect its rooms to be noisy.
This knowledge is specified in Line~1--5 of the following program:
%
\lstinputlisting[xrightmargin=-26pt]{examples/opt.lp}
%
Line~6--8 contribute optimization statements in inverse order of significance,
according to which we want to choose the best hotel to book.
The most significant optimization statement in Line~8 states that
avoiding noise is our main priority.
The secondary optimization criterion in Line~7 consists of
minimizing the cost per star.
Finally, the third optimization statement in Line~6 specifies that we want
to maximize the number of stars among hotels that are otherwise indistinguishable.
The optimization statements in Line~6--8 are instantiated as follows:%
\marginlabel{%
  The full ground program is obtained by invoking:\\
  \code{\mbox{~}gringo -t \textbackslash\\\mbox{~}examples/opt.lp}}
%
\begin{lstlisting}[firstnumber=6,breaklines,breakindent=48pt]
maximize [ hotel(1) = 5, hotel(2) = 4, hotel(3) = 3,    hotel(4) = 3, hotel(5) = 2 ].
minimize [ hotel(1) = 34, hotel(2) = 35, hotel(3) = 30,   hotel(4) = 25, hotel(5) = 30 ].
minimize [ noisy = 1 ].
\end{lstlisting}
If we now use \clasp\ to compute an optimal answer set,%
\marginlabel{%
  To compute the unique optimal answer set, invoke:\\
  \code{\mbox{~}gringo \textbackslash\\
        \mbox{~}examples/opt.lp | \textbackslash\\
        \mbox{~}clasp -n 0}\\
  or alternatively:\\
  \code{\mbox{~}clingo -n 0 \textbackslash\\
        \mbox{~}examples/opt.lp}}
we find that hotel~\const{4} is not eligible because it implies \pred{noisy}.
Thus, hotel~\const{3} and~\const{5} remain as optimal w.r.t.\ the second most
significant optimization statement in Line~7.
This tie is broken via the least significant optimization statement in Line~6
because hotel~\const{3} has one star more than hotel~\const{5}.
We thus decide to book hotel~\const{3} offering~\const{3} stars
to cost~\const{90} per night.
\eexample
\end{example}

\begin{newstuff}
	\begin{itemize}
		\item \#'s are missing again
		\item priorities can be attatched
			\begin{itemize}
				\item low/high integer corresponds to low/high priority
				\item if no priority given, the n-th minimize/maximize gets priority n
					  (for conformity with lparse)
			\end{itemize}
	\end{itemize}
\end{newstuff}

\subsubsection{Meta-Statements}\label{subsec:gringo:meta}

After considering the language of logic programs,
we now introduce features going beyond the contents of a program.

\paragraph{Comments.}
To keep records of the contents of a logic program,
a logic program file may include comments.
A comment until the end of a line is initiated by symbol ``\code{\%},''
and a comment within one or over multiple lines is enclosed
by ``\code{\%*}'' and ``\code{*\%}.''
As an abstract example, consider:
%
\begin{lstlisting}[numbers=none,escapechar=@]
@\textit{logic program}@  %* enclosed comment *%  @\textit{logic program}@
@\textit{logic program}@  % comment till end of line
@\textit{logic program}@
%*
comment over multiple lines
*%
@\textit{logic program}@
\end{lstlisting}

\paragraph{Hiding Predicates.}
Sometimes, one may be interested only in a subset of the atoms belonging
to an answer set.
In order to suppress the atoms of ``irrelevant'' predicates from the output,
the \code{\#hide} declarative (in which~\code{\#} is optional) can be used.
The meanings of the following statements are indicated via accompanying comments:
%
\begin{lstlisting}[numbers=none]
#hide.          % Suppress all atoms in output
 hide.          % Same as "#hide."
#hide p/3.      % Suppress all atoms of predicate p/3 in output
 hide p/3.      % Same as "#hide p/3."
#hide p(X,Y,Z). % Same as "#hide p/3."
 hide p(X,Y,Z). % Same as "#hide p/3."
\end{lstlisting}
%
In order to selectively include the atoms of a certain predicate in the output,
one may use the \code{\#show} declarative (in which~\code{\#} is again optional).
Here are some examples:
%
\begin{lstlisting}[numbers=none]
#show p/3.      % Include all atoms of predicate p/3 in output
 show p/3.      % Same as "#show p/3."
#show p(X,Y,Z). % Same as "#show p/3."
 show p(X,Y,Z). % Same as "#show p/3."
\end{lstlisting}
%
A typical usage of \code{\#hide} and \code{\#show} is to hide all predicates
via ``\code{\#hide.}'' and to selectively re-add atoms of certain predicates
\code{p/$n$} to the output via ``\code{\#show p/$n$.}''

\begin{newstuff}
	\begin{itemize}
		\item \#hide/show statements in predicate form have a new meaning now
		\item \#hide p(X,X) will only hide p(1,1), p(2,2), p(a,a)
		\item they can also have conditions now
	\end{itemize}
\end{newstuff}

\begin{newstuff}
\paragraph{External Statements.}
\com{write me!!!}
\end{newstuff}

\paragraph{Constant Replacement.}
Constants appearing in a logic program may actually be placeholders for
concrete values to be provided by a user.
An example of this will be given in Section~\ref{subsec:ex:color}.
Via the \code{\#const} declarative (\code{\#}~is optional),
one may define a default value to be inserted for a constant.
Such a default value can still be overridden via command line option
\code{--const} (cf.\ Section~\ref{subsec:opt:gringo}).
Syntactically, \code{\#const} must be followed by an assignment having
a (symbolic) constant on the left-hand side and a term on the right-hand side.
Some exemplary \code{\#const} declarations are:
%
\begin{lstlisting}[numbers=none]
#const x = 42.
 const y = f(g,h).
\end{lstlisting}

\begin{newstuff}
	\begin{itemize}
		\item for performance reasons \#const statements are evaluated in order of appearance now
	\end{itemize}
\end{newstuff}

\paragraph{Domain Declarations.}
Usually, variable names are local to a rule,
where they must be bound via appropriate atoms (cf.\ Section~\ref{subsec:lambda}).
This locality can be undermined by using
\code{\#domain} declarations (\code{\#}~is optional)
that globally associate variable names to atoms.
An associated atom is then simply added to the body of a rule in which
such a predefined variable name occurs in.
The following is a made-up example:
%
\begin{lstlisting}[numbers=none]
p(1,1). p(1,2).
#domain p(X,Y).
 domain p(Y,Z).
q(Z,X) :- not p(Z,X).
\end{lstlisting}
%
The above program is a priori not level-restricted because
variables~\var{X} and~\var{Z} are unbound in the last rule.
However, as they belong to \code{\#domain} declarations,
\gringo\ and \clingo\ expand the last rule to:
%
\begin{lstlisting}[numbers=none]
q(Z,X) :- p(X,Y), p(Y,Z), not p(Z,X).
\end{lstlisting}
%
Observe that the resulting program is level-restricted (and stratified).
\begin{newstuff}
	\begin{itemize}
		\item they are still there don't use them they are evil
	\end{itemize}
\end{newstuff}

\paragraph{Compute Statements.}
These statements are artefacts supported for backward compatibility.
Although we strongly recommend to avoid compute statements,
we now describe their syntax.
A compute statement is of the form
``$\const{\#compute}~n \code{\char`\{}\dots\code{\char`\}}\code{.}$''
(\code{\#}~and non-negative integer $n$ are optional),
where the ``$\code{\char`\{}\dots\code{\char`\}}$'' part
is similar to a \const{count} aggregate.
The meaning is that all literals contained in
``$\code{\char`\{}\dots\code{\char`\}}$'' must hold w.r.t.\ answer sets
that are to be computed,
while~$n$ specifies a number of answer sets to compute.
As \clasp, \clingo, and \iclingo\ provide command line option
\code{--number} (cf.\ Section~\ref{subsec:opt:clasp})
to specify how many answer sets are to be computed,
they simply ignore~$n$.
Furthermore,
the ``$\code{\char`\{}\dots\code{\char`\}}$'' part can equivalently
be expressed in terms of integrity constraints,
as indicated in the comments provided along with the following example:
%
\begin{lstlisting}[xrightmargin=-20pt,numbers=none]
q(1;2).
{ p(1..5) }.
#compute 0 { p(X) : q(X) }.       % :- 1 { not p(X) : q(X) }.
 compute   { not p(X) : X=4..5 }. % :- 1 { p(X) : X=4..5 }.
\end{lstlisting}
\begin{newstuff}
	\begin{itemize}
		\item not yet there don't use them they are evil
	\end{itemize}
\end{newstuff}

\begin{newstuff}
\subsubsection{Lua Support}\label{subsec:lang:lua}
\com{write me!}
\end{newstuff}

\subsection{Input Language of \iclingo}\label{subsec:lang:iclingo}

System \iclingo~\cite{gekakaosscth08a} extends \clingo\ by an
\emph{incremental} computation mode that incorporates both grounding and solving.
Hence, its input language includes all constructs described
in Section~\ref{subsec:lang:gringo}.%
\footnote{%
  In its current version, \iclingo\ has no built-in support for
  classical negation (cf.\ Section~\ref{subsec:gringo:negation}).
  That is, complementary atoms may jointly belong to answer sets,
  unless explicitly prohibited by integrity constraints.}
In addition, \iclingo\ deals with statements of the following form:
%
\begin{lstlisting}[numbers=none,escapechar=@]
#base.
#cumulative @\textit{constant}@.
#volatile   @\textit{constant}@.
\end{lstlisting}
%
Via ``\code{\const{\#base}.},''
the subsequent part of a logic program is declared as static, that is,
it is processed only once at the beginning of an incremental computation.
In contrast, ``\code{\const{\#cumulative} \textit{constant}.}''
and ``\code{\const{\#volatile} \textit{constant}.}'' are used to declare
a (symbolic) \code{\textit{constant}} as a placeholder for incremental step numbers.
In the parts of a logic program below a \const{\#cumulative} statement,
\code{\textit{constant}} is in each step replaced with the current step number,
and the resulting rules, facts, and integrity constraints are accumulated
over a whole incremental computation.
While the replacement of \code{\textit{constant}} is similar,
a logic program part below a \const{\#volatile} statement is local to steps,
that is, all rules, facts, and integrity constraints computed in one step
are dismissed before the next incremental step.
Note that the type of a logic program part (static, cumulative, or volatile)
is determined by the last
\const{\#base}, \const{\#cumulative}, or \const{\#volatile} statement
preceding it.

During an incremental computation, all static program parts are grounded first,
while cumulative and volatile parts are grounded step-wise,
replacing \code{\textit{constant}}s with successive step numbers starting from~\const{1}.
After a grounding step, \clasp\ is usually invoked via an internal interface
(like with \clingo), and the incremental computation stops after a step
in which at least one answer set has been found by \clasp.
This default behavior can be readapted via command line 
options (cf.\ Section~\ref{subsec:opt:iclingo}).
For obtaining a well-defined incremental computation result,
it is important that (ground) head atoms within static, cumulative, and volatile program parts
are distinct from each other, and they must also be different from step to step
(see~\cite{gekakaosscth08a} for details).
In Section~\ref{subsec:ex:block},
we will provide a typical example in which these conditions naturally hold.

\begin{newstuff}
	\begin{itemize}
		\item rules won't be instantiated using the same (global) substitution twice
		\item :- \{ p(X) \}. in cumulative part won't be regrounded even though domain of p/1 changed
	\end{itemize}
\end{newstuff}
\subsection{Input Language of \clasp}\label{subsec:lang:clasp}

Solver \clasp~\cite{gekanesc07b} works on logic programs in \lparse's
output format~\cite{lparseManual}.
This numerical format, which is not supposed to be human-readable,
is output by \gringo\ and can be piped into \clasp.
Such an invocation of \clasp\ looks as follows:
%
\begin{lstlisting}[numbers=none]
gringo [ options | filenames ] | clasp [ number | options ]
\end{lstlisting}
%
Note that \code{number} may be provided to specify a maximum number of answer sets
to be computed, where~\code{0} makes \clasp\ compute all answer sets.
This maximum number can also be set via
option \code{--number} or its abbreviation \code{-n}
(cf.\ Section~\ref{subsec:opt:clasp}).
By default, \clasp\ computes one answer set (if it exists).
If a logic program in \lparse's output format has been stored in a \code{file},
it can be redirected into \clasp\ as follows:%
\footnote{%
  The same is achieved by using option \code{--file} or its short form \code{-f}
  (cf.\ Section~\ref{subsec:opt:clasp}).}
%
\begin{lstlisting}[numbers=none]
clasp [ number | options ] < file
\end{lstlisting}
%
Via option \code{--dimacs},
\clasp\ can also be instructed to compute models of a propositional formula
in DIMACS/CNF format~\cite{dimacs}.
If such a formula is contained in \code{file},
then \clasp\ can be invoked in the following way:
%
\begin{lstlisting}[numbers=none]
clasp [ number | options ] --dimacs < file
\end{lstlisting}
%
Finally, \clasp\ may be used as a library, as done within \clingo\ and \iclingo.


%%% Local Variables: 
%%% mode: latex
%%% TeX-master: "guide"
%%% End: 
