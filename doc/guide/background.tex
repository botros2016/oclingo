\section{Background}\label{sec:background}

In Section~\ref{subsec:asp},
we give a brief introduction to Answer Set Programming,
which constitutes the basic framework for the tools we describe here.
This framework deals with logic programs,
whose syntax and semantics are formally introduced in
Section~\ref{subsec:syntax} and~\ref{subsec:semantics}, respectively.
We invite the reader to merely skim or even skip these two sections
upon the first reading, and to rather look them up later on if
formal details are requested.
However, we want to stress that the syntax we use for logic programs
admits (uninterpreted) function symbols with non-zero arity,
whose full support by our tools is rather exceptional and thus innovative.


\begin{figure}[tb]
\centering
\begin{picture}(300,120)(-150,-60)
\put(-80,+40){\makebox(0,0){\framebox(80,20){Problem}}}
\put(-80,-40){\makebox(0,0){\framebox(80,20){Logic Program}}}
\put(+80,+40){\makebox(0,0){\framebox(80,20){Solution(s)}}}
\put(+80,-40){\makebox(0,0){\framebox(80,20){Answer set(s)}}}
\put(-80,+30){\vector(0,-1){60}}
\put(-40,-40){\vector(+1,0){80}}
\put(+80,-30){\vector(0,+1){60}}
\put(-120,  0){\makebox(0,0){{Representation}}}
\put(+120,  0){\makebox(0,0){{Interpretation}}}
\put(   0,-55){\makebox(0,0){{Computation}}}
\end{picture}
\caption{Declarative Problem Solving in ASP.\label{fig:asp}}
\end{figure}

%%% Local Variables: 
%%% mode: latex
%%% TeX-master: "../guide"
%%% End: 

\subsection{Answer Set Programming}\label{subsec:asp}

Answer Set Programming (ASP)
\cite{ankolisc05a,baral03a,gelleo02a,lifschitz02a,martru99a,niemela99a}
emerged in the late 1990s as a declarative paradigm
for modeling and solving search problems.
As illustrated in Figure~\ref{fig:asp},
the basic approach of ASP is to represent a problem as a logic program~$\Pi$
such that particular Herbrand models of~$\Pi$, called answer sets,
correspond to problem solutions.
In contrast to traditional logic programming languages,
e.g., Prolog~\cite{nilmal95a},
ASP programs are sets (not lists) of rules,
and computations are oriented towards models (not proofs).
Let us illustrate this on an example.

\begin{example}\label{ex:flies}
Consider a logic program comprising the following facts:
%
\lstinputlisting{examples/bird.lp}
% \listinginput{1}{examples/bird.lp}
%
These facts express that constants \const{tux} and \const{tweety}
stand for birds.
Furthermore, \const{tux} is a penguin, 
and \const{tweety} is a chicken.
%
Note that the \emph{unique names assumption} applies, that is,
$\const{tux}\neq\const{tweety}$ holds.
Besides the two constants,
the language of the above facts contains predicates
\pred{bird}/$1$, \pred{penguin}/$1$, and \pred{chicken}/$1$
(each having arity~$1$).
%
We next augment the above facts with the following rules:%
\marginlabel{%
  Instead of \code{neg\mus flies(X)}, one could write
  \code{-flies(X)} to make explicit that
  instances (obtained for~\var{X}) amount to the classical (or strong) negation of
  \code{flies(X)}.}
%
\lstinputlisting[firstnumber=3]{examples/fly.lp}
%
Informally, such rules correspond to implications where the
left hand side of connective ``\code{:-}'' (called head)
is derived if all premises on the right hand side
(their conjunction called body) hold.
Connective ``\code{not}'' stands for \emph{default negation},
and uppercase letter~\var{X} is the name of a first-order variable.
Variables are local to rules, that is,
a variable name refers to different variables in distinct rules.
Finally, every variable is universally quantified and thus
applies to all terms in the language of a logic program.

The rule in Line~3 formalizes that any instance of~\var{X} that is
a bird \pred{flies}, as long as \pred{neg\mus flies} does not hold.
The converse relationship between \pred{neg\mus flies} and \pred{flies}
is expressed in Line~4.
At this point, observe that, though syntactically correct,
the above program ``makes no sense'' in Prolog because a query%\hspace{\listingoffset}
% \begin{alltt}
% ?- flies(tweety).
% \end{alltt}
\begin{lstlisting}[numbers=none]
?- flies(tweety).
\end{lstlisting}
can be answered neither positively nor negatively.
In ASP, however, the rules in Line~3 and~4 admit multiple belief states,
depending on whether \pred{flies} or \pred{neg\mus flies} is assumed for a bird.
In addition, the rule in Line~5 requires \pred{neg\mus flies} to hold for every penguin.
Combining this knowledge with the facts in Line~1 and~2
makes us derive \code{\pred{neg\mus flies}(\const{tux})},
and it provides us with two alternatives for \const{tweety}:
either \code{\pred{flies}(\const{tweety})} or \code{\pred{neg\mus flies}(\const{tweety})} holds.
These alternatives cannot jointly hold because, under such an assumption,
neither of them is derived, thus,
violating the justification principle of ASP (cf.\ Section~\ref{subsec:semantics}).
Without already going into formal details,
we conclude that the above program has two answer sets:
one according to which \const{tweety} flies and one where \const{tweety} does not fly.
\eexample
\end{example}

\begin{figure}[tb]
\centering
\begin{picture}(300,120)(-100,-60)
\put(-80,+40){\makebox(0,0){\framebox(80,20){Logic Program}}}
\put(-80,-40){\oval(80,20)}\put(-99,-43){Grounder}
\put(+50,-40){\makebox(0,0){\framebox(80,20){Ground Program}}}
\put(+180,+40){\makebox(0,0){\framebox(80,20){Answer set(s)}}}
\put(+180,-40){\oval(80,20)}\put(+167,-43){Solver}
\put(-80,+30){\vector(0,-1){60}}
\put(-40,-40){\vector(+1,0){50}}
\put(+90,-40){\vector(+1,0){50}}
\put(+180,-30){\vector(0,+1){60}}
\end{picture}
\caption{Basic Architecture of ASP Systems.\label{fig:system}}
\end{figure}

%%% Local Variables: 
%%% mode: latex
%%% TeX-master: "../guide"
%%% End: 

The computation of answer sets usually consists of two phases.
First, a \emph{grounder} substitutes the variables in a logic program~$\Pi$
by variable-free terms, resulting in a propositional program~$\alt{\Pi}$.
Second, the answer sets of~$\alt{\Pi}$ are computed by a \emph{solver}.
This prototypical architecture is visualized in Figure~\ref{fig:system}.
Of course, a ground program~$\alt{\Pi}$ must be finite and equivalent to
input program~$\Pi$.
To this end, grounders like \lparse~\cite{lparseManual},
the one embedded into \dlv~\cite{dlv03a},
and \gringo\footnotemark~\cite{gescth07a}
impose certain restrictions on~$\Pi$ (cf.\ Section~\ref{subsec:lambda}).
An obtained ground program~$\alt{\Pi}$ can be piped into a solver, such as 
\assat~\cite{linzha04a}, 
\clasp\footnotemark[\value{footnote}]~\cite{gekanesc07b},
\cmodels~\cite{gilima06a},
\nomorepp~\cite{angelinesc05c},
\smodels~\cite{siniso02a}, or
\smodelscc~\cite{warsch04a},
in order to compute answer sets of~$\alt{\Pi}$, matching the ones of~$\Pi$.
All of the aforementioned solvers deal with \lparse's output format,
a numerical representation of~$\alt{\Pi}$,
while \dlv\ couples its grounding and its solving component directly
via an internal interface,
an approach also realized by \clingo\footnotemark[\value{footnote}].
Finally, \iclingo\footnotemark[\value{footnote}]~\cite{gekakaosscth08a} integrates and interleaves
grounding and solving within incremental computations.
%
%\addtocounter{footnote}{-1}%
\footnotetext{Described in this guide.}

The next example illustrates the computation of answer sets for
the logic program given in Example~\ref{ex:flies}.

\begin{example}\label{ex:flies:ground}
Reconsidering the facts in Line~1 and~2 of Example~\ref{ex:flies},
we identify two constants: \const{tux} and \const{tweety}.
They can be substituted for variables~\var{X} in the rules in Line~3--5.
In fact, Line~1--5 of Example~\ref{ex:flies} are a shorthand for the ground instantiation:
%
\lstinputlisting[nolol]{examples/bird.lp}
\lstinputlisting[firstnumber=3,xrightmargin=-8pt]{examples/bfall.lp}
%
Note that the facts in Line~1 and~2 are unaltered from Example~\ref{ex:flies},
while the ground rules in Line~3--8 are obtained by instantiating variables~\var{X}.
We have that the answer sets of the above ground program match
those of the original input program in Example~\ref{ex:flies}.

In practice, grounders try to avoid producing the full ground instantiation
of an input program by applying answer-set preserving simplifications.
In particular, facts can (recursively) be eliminated.
Thus, \gringo\ computes the following ground program:%
\marginlabel{%
  The ground program in text format is obtained via call:\\
  \code{\mbox{~}gringo -t \textbackslash\\
        \mbox{~}examples/bird.lp \textbackslash\\
        \mbox{~}examples/fly.lp}}
%
\lstinputlisting[nolol]{examples/bird.lp}
\lstinputlisting[numbers=none]{examples/bfeff.lp}
%
Observe that \code{\pred{neg\mus flies}(\const{tux})} has become a fact because
\code{\pred{penguin}(\const{tux})} is known.
Similarly, \code{\pred{bird}(\const{tux})} and \code{\pred{bird}(\const{tweety})}
have been removed from bodies of rules, making use of the fact that ``\textit{true} \& something'' is
equivalent to ``something.''
Applying the complementary principle that ``\textit{false} \& something'' evaluates to ``\textit{false},''
the rule with \code{not \pred{neg\mus flies}(\const{tux})} in the body (Line~3) has been dropped completely.
After performing these simplifications, the ground program computed by \gringo\ is still
equivalent to the input program in Example~\ref{ex:flies}, viz.,
there is one answer set such that \const{tweety} flies and one where \const{tweety} does not fly.
In fact, running \clasp\ yields:%
\marginlabel{%
  The two answer sets are computed by piping the output of \gringo\
  into \clasp:\\
  \code{\mbox{~}gringo \textbackslash\\
        \mbox{~}examples/bird.lp \textbackslash\\
        \mbox{~}examples/fly.lp | \textbackslash\\
        \mbox{~}clasp -n 0}\\
  Note that \gringo\ is called without option \code{-t},
  while option \code{-n 0} makes \clasp\ compute all answer sets
  (rather than just one, as done by default).
\\
  Alternatively, one may use \clingo\ to compute the answer sets.
  To this end, invoke:\\
  \code{\mbox{~}clingo -n 0 \textbackslash\\
        \mbox{~}examples/bird.lp \textbackslash\\
        \mbox{~}examples/fly.lp}}
%
\begin{lstlisting}[numbers=none,escapechar=@]
Answer: 1
bird(tux) penguin(tux) bird(tweety) chicken(tweety) @\textbackslash@
neg_flies(tux) flies(tweety)
Answer: 2
bird(tux) penguin(tux) bird(tweety) chicken(tweety) @\textbackslash@
neg_flies(tux) neg_flies(tweety)
\end{lstlisting}
%
Note that the order of the answer sets is incidental and 
not obliged to the order of rules and bodies in the input,
as the semantics of ASP programs is purely declarative.
\eexample
\end{example}

We have seen how an ASP system computes answer sets of a search problem
represented by a set of facts (Line~1 and~2 in Example~\ref{ex:flies})
and a set of schematic rules with first-order variables
(Line~3--5 in Example~\ref{ex:flies}).
This separation of a problem into a knowledge part, called \emph{encoding},
and a data part, called \emph{instance}, is not a coincidence
but a common methodology in ASP \cite{martru99a,niemela99a,schlipf95a},
sometimes called \emph{uniform} definition.
In fact, the possibility to describe problems in a uniform way is
a major advantage of ASP, as it promotes simplicity and flexibility in
knowledge representation.
Together with the availability of efficient reasoning engines,
this makes ASP an attractive and powerful paradigm
for declarative problem solving.
By explaining and illustrating the functionalities of
grounder \gringo, solver \clasp, and their bondings in \clingo\ and \iclingo,
this guide aims at enabling the reader to make (better) use of ASP.


\subsection{Syntax of Logic Programs}\label{subsec:syntax}

This section gives a brief account of the formal syntax of logic programs,
needed for defining their semantics.
Consult, e.g., \cite{lloyd87a} for a detailed introduction.
The language of a logic program is composed from sets
%
\begin{itemize}
\item $\mathcal{P}=\{p_1/i_1,\dots,p_m/i_m\}$ of \emph{predicates}
      (with arities $i_1,\dots,i_m$),
\item $\mathcal{F}=\{f_1/j_1,\dots,f_n/j_n\}$ of \emph{functions}
      (with arities $j_1,\dots,j_n$), and
\item $\mathcal{V}=\{V_1,V_2,V_3,\dots\}$ of \emph{variables}
\end{itemize}
along with connectives (``\code{:-},'' ``\code{,},'' ``\code{not},'' etc.) and
punctuation symbols (``\code{(},'' ``\code{)},'' ``\code{.},'' etc.).
A function~$f/j$ is called a \emph{constant} if $j=0$, that is,
if~$f$ has no arguments.
The set~$\mathcal{T}$ of \emph{terms} is the smallest set containing~$\mathcal{V}$
and all expressions $f(t_1,\dots,t_k)$ such that $f/k$ is a function 
from~$\mathcal{F}$ and $t_1,\dots,t_k$ are terms in~$\mathcal{T}$.
Note that every variable and every constant is an elementary term,
while functions with non-zero arity induce compound terms.
By $\ground{\mathcal{T}}$, we denote the set of all variable-free
terms in~$\mathcal{T}$.
Note that $\ground{\mathcal{T}}$ is infinite
as soon as~$\mathcal{F}$ contains at least one constant and one function
with non-zero arity.
The set~$\mathcal{A}$ of \emph{atoms} consists of 
$\top$ (``\textit{true}''), $\bot$ (``\textit{false}''), and all expressions
$p(t_1,\dots,t_k)$ such that $p/k$ is a predicate
from~$\mathcal{P}$ and $t_1,\dots,t_k$ are terms in~$\mathcal{T}$.
As with terms, $\ground{\mathcal{A}}$ denotes the set of all variable-free
atoms in~$\mathcal{A}$.

A \emph{rule}~$r$ is an expression of the form $\mathit{Head}~\code{:-}~\mathit{Body}\code{.}$,
where $\mathit{Head}$ and $\mathit{Body}$ are composed from atoms in $\mathcal{A}$
(other than $\bot$ and $\top$), punctuation symbols, and connectives other than ``\code{:-}.''
The intuitive reading of a rule is that $\mathit{Head}$ follows from $\mathit{Body}$,
and either $\mathit{Body}$ or $\mathit{Head}$ may be omitted, in which case the
rule is called a \emph{fact} or an \emph{integrity constraint}, respectively.
A fact is written in the form $\mathit{Head}\code{.}$ and understood as 
$\mathit{Head}~\code{:-}~\top\code{.}$, and an integrity constraint
$\code{:-}~\mathit{Body}\code{.}$ is a shorthand for $\bot~\code{:-}~\mathit{Body}\code{.}$
By $\vars{r}$, we denote the set of all variables in rule~$r$,
and~$r$ is called \emph{ground} if $\vars{r}=\emptyset$.
A \emph{ground substitution} is a mapping $\sigma:\vars{r}\rightarrow\ground{\mathcal{T}}$,
and~$r\sigma$ is the ground rule obtained by replacing every occurrence of a
variable $V\in\vars{r}$ with~$\sigma(V)$.
By $\ground{r}$, we denote the set of ground rules~$r\sigma$
obtained from~$r$ by applying all possible ground substitutions~$\sigma$.
A \emph{logic program}~$\Pi$ is a set of rules,
and the \emph{ground instantiation} of~$\Pi$ is
$\ground{\Pi}=\bigcup_{r\in\Pi}\ground{r}$.
By convention, we assume that~$\mathcal{P}$ and~$\mathcal{F}$ are the sets
of all predicates and functions, respectively, that occur in~$\Pi$.
Note that $\ground{\Pi}$ is infinite if
there is at least one rule $r\in\Pi$ such that $\vars{r}\neq\emptyset$ and 
if~$\mathcal{F}$ contains at least one constant and one function
with non-zero arity.

For a well-known class of logic programs, called \emph{normal},
rules are of the form:
\begin{equation}\label{eq:normal:rule}
  A_0~\code{:-}~A_1\code{,}\dots\code{,}A_m\code{,}
  \code{not}~A_{m+1}\code{,}\dots\code{,}\code{not}~A_n\code{.}
\end{equation}
%
For $0\leq k\leq n$, every~$A_k$ is an atom, that is, the head is an atom and
the body is a (possibly empty) conjunction of atoms that may be preceded by ``\code{not}.''
As mentioned in Section~\ref{subsec:asp},
the order of elements in the body is not essential,
so that they can be regrouped to match the form in~(\ref{eq:normal:rule}).
For a rule~$r$ as in~(\ref{eq:normal:rule}), we define 
$\head{r}=A_0$, $\pbody{r}=\{A_1,\dots,A_m\}$, and $\nbody{r}=\{A_{m+1},\dots,A_n\}$.
If $r=A_0\code{.}$ is a fact, then
$\pbody{r}=\nbody{r}=\nolinebreak\emptyset$.
The particular structure of normal programs admits a rather intuitive 
characterization of their semantics in the next section.

\begin{example}\label{ex:syntax}
The language of the normal program in Example~\ref{ex:flies} includes
predicates
\(
\mathcal{P}=
\{
\pred{bird}/1,\pred{penguin}/1,\pred{chicken}/1,
\pred{flies}/1,\pred{neg\mus flies}/1\}
\)
and functions
\(
\mathcal{F}=
\{
\const{tux}/0,\const{tweety}/0
\}
\).
As the arity of the latter is~$0$,
terms \const{tux} and \const{tweety} are constants.
By substituting them for variables,
we obtain the ground instantiation given in Example~\ref{ex:flies:ground}.
For instance, the ground rules
%
\begin{lstlisting}[firstnumber=3]
flies(tux)    :- bird(tux),    not neg_flies(tux).
flies(tweety) :- bird(tweety), not neg_flies(tweety).
\end{lstlisting}
%
are obtained from:
%
\begin{lstlisting}[numbers=none]
flies(X) :- bird(X), not neg_flies(X).
\end{lstlisting}
%
Observe that the three occurrences of variable name~\var{X}
stand for a single variable, so that substitutions
$\{\var{X}\mapsto\const{tux}\}$ and $\{\var{X}\mapsto\const{tweety}\}$
induce the above ground rules.
Notably, sets~$\mathcal{P}$ and~$\mathcal{F}$ of predicates and functions
need not be disjoint because atoms and terms are also distinguished
by the context they occur in,
and a predicate or function symbol may be used with different arities.
For instance, we could add a rule
%
\begin{lstlisting}[numbers=none]
flies :- flies(X).
\end{lstlisting}
%
over predicates \pred{flies}/$0$ and \pred{flies}/$1$. 
In fact, such rules are not unusual to express that there is some object
with certain properties, while it does not matter which object it is.
Finally, note that an infinite ground instantiation would result from adding rule
\begin{lstlisting}[numbers=none]
bird(parent(X)) :- bird(X).
\end{lstlisting}
in view of function \const{parent}/$1$ with non-zero arity.
\eexample
\end{example}


\subsection{Semantics of Logic Programs}\label{subsec:semantics}

The semantics of logic programs is given by answer sets~\cite{gellif91a},
also called stable models~\cite{gellif88a}.
In view of the rich input language of \gringo\ (cf.\ Section~\ref{subsec:lang:gringo}),
we below recall a definition of answer sets
for propositional theories~\cite{ferraris05a} and
explain the semantics of logic programs by translation
into propositional logic.
We will also provide an alternative direct definition of answer sets
for the class of normal programs.

We consider propositional formulas composed from atoms, $\bot$,
and connectives~$\wedge$, $\vee$, and~$\rightarrow$.
Furthermore, $\top$ and~$\neg F$ are used as shorthands for
$(\bot\rightarrow\bot)$ and $(F\rightarrow\bot)$, respectively.
The \emph{reduct}~$\reduct{F}{X}$ of a propositional formula~$F$
relative to a set~$X$ of atoms is defined recursively as follows:
\begin{itemize}\addtolength{\itemsep}{-3pt}
\item $\reduct{F}{X}=\bot$ ~if $X\not\models F$,
\item $\reduct{F}{X}=F$ ~if $F\in X$, and
\item $\reduct{F}{X}=(\reduct{G}{X}\circ\reduct{H}{X})$ ~if 
      $X\models F$ and $F=(G\circ H)$ for $\circ\in\{\wedge,\vee,\rightarrow\}$.%
\footnote{$\models$ is the standard satisfaction relation of propositional logic
  between interpretations~$X$ and formulas~$F$.}
\end{itemize}
Essentially, the reduct relative to~$X$ is obtained by replacing
all maximal unsatisfied subformulas of~$F$ with~$\bot$.
As a matter fact, this implies that any maximal negative subformula 
of the form $(G\rightarrow \bot)$ is replaced by either $\bot$ (if $X\models G$)
or $(\bot\rightarrow\bot)=\top$ (if $X\not\models G$).
Furthermore, any atom occurring in $\reduct{F}{X}$ belongs to~$X$.

A propositional theory~$\Phi$ is a set of propositional formulas,
and the reduct of~$\Phi$
relative to a set~$X$ of atoms is
$\reduct{\Phi}{X}=\{\reduct{F}{X} \mid F\in\Phi\}$.
Furthermore, $X$ is a \emph{model} of~$\Phi$ if $X\models F$
for every $F\in\Phi$.
Finally, $X$ is an \emph{answer set} of~$\Phi$ if
$X$ is a $\subseteq$-minimal model of~$\reduct{\Phi}{X}$, that is,
if $X$ is a model of~$\Phi$ such that no $Y\subset X$
is a model of~$\reduct{\Phi}{X}$.
The latter condition uses the fact that~$X$ is a model of~$\reduct{\Phi}{X}$
iff $X$ is a model of~$\Phi$.%
\footnote{If~$X$ is not a model of~$\Phi$, then 
  there is some $F\in\Phi$ such that $X\not\models F$,
  which in turn implies $\bot\in\reduct{\Phi}{X}$.}
Also note that, if~$X$ is an answer set of~$\Phi$,
it is the unique $\subseteq$-minimal model of~$\reduct{\Phi}{X}$
because all atoms occurring in~$\reduct{\Phi}{X}$ belong to~$X$
(which excludes incomparable $\subseteq$-minimal models).
This must not be confused
with uniqueness of an answer set of~$\Phi$.
In fact, $\Phi$ may have zero, one,
or multiple answer sets~$X$ (being $\subseteq$-minimal models of
distinct reducts~$\reduct{\Phi}{X}$).
In a sense, the $\subseteq$-minimality of an answer set~$X$ as a 
model of~$\reduct{\Phi}{X}$ stipulates the atoms in~$X$ to be justified by~$\Phi$
(or necessarily true) under the assumption of~$X$.

For a logic program~$\Pi$, answer sets of~$\Pi$ can now be explained
by translation to a propositional theory~$\Phi[\Pi]$.
To this end, let
\begin{itemize}
\item $\Phi[\Pi] = \bigcup_{r\in\ground{\Pi}}\phi[r]$,
\item $\phi[\mathit{Head}~\code{:-}~\mathit{Body}\code{.}] = 
       (\phi[\mathit{Body}]\rightarrow\psi[\mathit{Head}])$,
\item $\phi[A]=\psi[A]=A$ ~if $A\in\mathcal{A}$,
\item $\phi[\top]=\top$,%
\footnote{Recall that a fact $\mathit{Head}\code{.}$ is a shorthand for
          $\mathit{Head}~\code{:-}~\top\code{.}$}
\item $\phi[\code{not}~F] = \neg \phi[F]$, and
\item $\phi[G\code{,}H] = (\phi[G]\wedge\phi[H])$.
\end{itemize}
%
The translation~$\phi$ of a rule primarily consists of
straightforward syntactic conversions by replacing
``\code{:-},'' ``\code{,},'' and ``\code{not}'' with
$\rightarrow$, $\wedge$, and $\neg$, respectively.
However, it also contains a separate translation~$\psi$ for heads of rules,
which coincides with~$\phi$ on atoms by simply keeping them unchanged.
In Section~\ref{subsec:gringo:aggregate},
we will customize~$\psi$ in order to reflect the
``choice semantics'' for aggregates in heads of rules~\cite{ferlif05a,siniso02a}.
By virtue of the above translation,
we can simply define the answer sets of a logic program~$\Pi$
to be the answer sets of~$\Phi[\Pi]$.
We have thus specified answer sets for normal programs.

In order to provide more intuitions,
we now reproduce a direct definition of answer sets for normal programs.
This definition builds on the fact that the reduct for a normal program
is a set of Horn clauses, for which the $\subseteq$-minimal model (if it exists) can be constructed
systematically in a bottom-up manner~\cite{dowgal84a}.
To this end, for a normal program~$\Pi$ and a set $X\subseteq\ground{\mathcal{A}}$,
we define:
\begin{itemize}
\item $T^0_{\Pi,X} = \emptyset$ ~and
\item $T^{i+1}_{\Pi,X} = \{\head{r} \mid 
                        r\in\ground{\Pi},
                        \pbody{r}\subseteq T^i_{\Pi,X},
                        \nbody{r}\cap X=\emptyset\}$.
\end{itemize}
It is not hard to check that operator $T_{\Pi,X}$ is monotonic,
as it adds head atoms of ground rules to its previous result if the positive
body atoms have been derived,
while negative bodies are merely evaluated w.r.t.~$X$.
Using this operator, we can equivalently define $X\subseteq\ground{\mathcal{A}}$
as an answer set of a normal program~$\Pi$ if $X=\bigcup_{0\leq i}T^i_{\Pi,X}$.
This equilibrium requirement exhibits two fundamental aspects of answer set semantics:
\begin{enumerate}
\item Negative conditions are first evaluated w.r.t.\ an answer set candidate.
\item The candidate must be justified by the result of the preceding evaluation.
\end{enumerate}
Provided that~$X$ is a model of~$\Pi$ (that is, of~$\Phi[\Pi]$),
the second item can be understood in the sense that all atoms of~$X$
must have a proof from normal program~$\Pi$ w.r.t.~$X$.

\begin{example}\label{ex:semantics}
The normal program in Example~\ref{ex:flies}
(or its ground instantiation given in Example~\ref{ex:flies:ground})
translates into the following propositional theory:
%
\begin{lstlisting}[escapechar=@]
bird(tux)     penguin(tux)
bird(tweety)  chicken(tweety) @\\@
bird(tux)   @\rlap{\,$\wedge$}@  @\rlap{\!$\neg$}@ neg_flies(tux)   @${}\rightarrow$@ flies(tux)
bird(tweety)@\rlap{\,$\wedge$}@  @\rlap{\!$\neg$}@ neg_flies(tweety)@${}\rightarrow$@ flies(tweety)
bird(tux)   @\rlap{\,$\wedge$}@  @\rlap{\!$\neg$}@ flies(tux)       @${}\rightarrow$@ neg_flies(tux)
bird(tweety)@\rlap{\,$\wedge$}@  @\rlap{\!$\neg$}@ flies(tweety)    @${}\rightarrow$@ neg_flies(tweety)
                 penguin(tux)   @${}\rightarrow$@ neg_flies(tux)
                 penguin(tweety)@${}\rightarrow$@ neg_flies(tweety)
\end{lstlisting}
%
Let
$X=\{
\code{\pred{bird}(\const{tux})},
\code{\pred{penguin}(\const{tux})},
\code{\pred{bird}(\const{tweety})},
\code{\pred{chicken}(\const{tweety})},\linebreak[1]
\code{\pred{neg\mus flies}(\const{tux})},
\code{\pred{flies}(\const{tweety})}
\}$.
Relative to $X$, we obtain the reduct:
%
\begin{lstlisting}[escapechar=@]
bird(tux)     penguin(tux)
bird(tweety)  chicken(tweety) @\\@
              @\rlap{\!$\bot$}@ @${}\rightarrow$@ @$\bot$@
bird(tweety)@\rlap{\,$\wedge$}@  @\rlap{\!$\top$}@ @${}\rightarrow$@ flies(tweety)
bird(tux)   @\rlap{\,$\wedge$}@  @\rlap{\!$\top$}@ @${}\rightarrow$@ neg_flies(tux)
              @\rlap{\!$\bot$}@ @${}\rightarrow$@ @$\bot$@
   penguin(tux)@${}\rightarrow$@ neg_flies(tux)
              @\rlap{\!$\bot$}@ @${}\rightarrow$@ @$\bot$@
\end{lstlisting}
%
We have that~$X$ is a model of the reduct (and the original program) because
the reduct does not contain formula~$\bot$ as an element.
In addition, observe that no proper subset of~$X$ is a model of the reduct.
From this, we conclude that~$X$ is an answer set.

The latter can also be verified by determining $\bigcup_{0\leq i}T^i_{\Pi,X}$
(for~$\Pi$ as in Example~\ref{ex:flies}):
\begin{tabular}{@{}r||l}
$i$ & $T^i_{\Pi,X}$
\\\hline\hline
$0$ &
---
\\\hline
$1$ &
$
\begin{array}{@{}l}
\code{\pred{bird}(\const{tux})},
\code{\pred{penguin}(\const{tux})},
\code{\pred{bird}(\const{tweety})},
\code{\pred{chicken}(\const{tweety})}
\end{array}
$
\\\hline
$2$  & 
$
\begin{array}{@{}l@{}}
\code{\pred{bird}(\const{tux})},
\code{\pred{penguin}(\const{tux})},
\code{\pred{bird}(\const{tweety})},
\code{\pred{chicken}(\const{tweety})},\\
\code{\pred{neg\mus flies}(\const{tux})},
\code{\pred{flies}(\const{tweety})}
\end{array}
$
\\\hline
${>\,}2$ &
\textit{do.}
\end{tabular}
\\

\noindent
For $i=1$, we simply derive the head atoms of facts.
They allow us to derive the remaining atoms of~$X$, viz.,
\code{\pred{neg\mus flies}(\const{tux})} and
\code{\pred{flies}(\const{tweety})}, in step~$i=\nolinebreak2$.
In fact, two rules produce \code{\pred{neg\mus flies}(\const{tux})},
while
\begin{lstlisting}[numbers=none]
flies(tweety) :- bird(tweety), not neg_flies(tweety).
\end{lstlisting}
is the unique rule with head \code{\pred{flies}(\const{tweety})} that ``fires.''
No further rules apply in step~$i=3$,
so that the set of atoms derived for~$i=2$ equals $\bigcup_{0\leq i}T^i_{\Pi,X}$.
We have thus reproduced~$X$ by bottom-up derivation,
which again shows that~$X$ is an answer set.
\eexample
\end{example}

We have above seen two definitions of answer sets that start
from an answer set candidate $X\subseteq\ground{\mathcal{A}}$,
which is then verified in some way.
Such definitions must not be confused with the computation of answer sets,
as (explicitly) enumerating all subsets of $\ground{\mathcal{A}}$
is infeasible in practice.
Rather, the computational schemes of solvers are similar
to the ``Davis-Putnam-Logemann-Loveland'' procedure (DPLL) \cite{dalolo62a,davput60}
or to ``Conflict-Driven Clause Learning'' (CDCL) \cite{marsak99a,mitchell05a,momazhzhma01a}.

%%% Local Variables: 
%%% mode: latex
%%% TeX-master: "guide"
%%% End: 
